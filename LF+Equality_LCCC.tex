\documentclass{article}
\usepackage[left=1in,right=1in,tmargin = 25mm,bmargin = 25mm]{geometry}
\usepackage[left=1in,right=1in]{geometry}
\usepackage{subfiles}
\usepackage{amsmath, amssymb, stmaryrd, verbatim, bbm} % math symbols
\usepackage{amsthm} % thm environment
\usepackage{mdframed} % Customizable Boxes
\usepackage{hyperref,nameref,cleveref,enumitem} % for references, hyperlinks
\usepackage[dvipsnames]{xcolor} % Fancy Colours
\usepackage{mathrsfs} % Fancy font
\usepackage{tikz, tikz-cd, float} % Commutative Diagrams
\usepackage{perpage}
\usepackage{parskip} % So that paragraphs look nice
\usepackage{ifthen,xargs} % For defining better commands
\usepackage{anyfontsize}
\usepackage[T1]{fontenc}
\usepackage[utf8]{inputenc}
\usepackage{tgpagella}
\usepackage{titlesec}
\usepackage{url}
\usepackage{listings}
\usepackage{stmaryrd}
\usepackage{proof}
\usepackage{multicol}
\usepackage{mathpartir}

% % Misc
\newcommand{\STLC}{\la^{\times \to}}
\newcommand{\LF}[1]{\la^{\Pi \mathsf{Eq}}[#1]}
\newcommand{\tdt}{\times \cdots \times}

% % Parens
\newcommand{\brkt}[1]{\left(#1\right)}
\newcommand{\sqbrkt}[1]{\left[#1\right]}
\newcommand{\bbrkt}[1]{\llbracket #1 \rrbracket}

% %
\newenvironment{intuit}{\textit{Intuition - }\em}{}

% % Types
\newcommand{\term}{\mathsf{term}}
\newcommand{\type}{\mathsf{type}}
\newcommand{\IsOf}[2]{#1 \, : \, #2}
\newcommand{\Subst}[3]{[#1 / #2] #3}
\newcommand{\SortCS}{\mathsf{Sort}}
\newcommand{\PiCS}[3]{\{#1 \, : \, #2\} \, #3}
\newcommand{\LamCS}[3]{[#1 \, : \, #2] \, #3}
\newcommand{\ApCS}[2]{#1 \, #2}
\newcommand{\EqCS}[3]{#2 =_{#1} #3}
\newcommand{\SelfCS}{\mathsf{self}}
\newcommand{\CtxExt}[3]{#1,#2 : #3}
\newcommand{\BNFSep}{\, \, | \, \,}
\newcommand{\CtxJdg}[1]{#1 \vdash \mathsf{ctx}}
\newcommand{\ClsJdg}[2]{#1 \vdash #2 \,\, \mathsf{cls}}
\newcommand{\ObjJdg}[3]{#1 \vdash #2 \, : \, #3 }
\newcommand{\IdClsJdg}[3]{\ClsJdg{#1}{#2 = #3}}
\newcommand{\IdObjJdg}[4]{\ObjJdg{#1}{#2 = #3}{#4}}
\newcommand{\CtxSortJdg}[1]{#1 \vdash \mathsf{Sort}}

% % Proof
\newenvironment{forward}{($\implies$)}{}
\newenvironment{backward}{($\limplies$)}{}

% % Ad-hoc Notation
\DeclareMathOperator{\supp}{supp}
\newcommand{\set}[1]{\left\{#1\right\}}
\newcommand{\st}{\,|\,}
\newcommand{\subs}{\subseteq}
\newcommand{\nothing}{\varnothing}
\newcommand{\inv}{^{-1}}
\renewcommand{\bar}[1]{\overline{#1}}
\newcommand{\abs}[1]{\left\vert #1 \right\vert}
\newcommand{\norm}[1]{\left\Vert #1 \right\Vert}
\newcommand{\<}{\langle}
\renewcommand{\>}{\rangle}

% % Greek
\newcommand{\al}{\alpha}
\newcommand{\be}{\beta}
\newcommand{\ga}{\gamma}
\newcommand{\de}{\delta}
\newcommand{\ep}{\varepsilon}
\newcommand{\io}{\iota}
\newcommand{\ka}{\kappa}
\newcommand{\la}{\lambda}
\newcommand{\om}{\omega}
\newcommand{\si}{\sigma}
\newcommand{\ze}{\zeta}

\newcommand{\Ga}{\Gamma}
\newcommand{\De}{\Delta}
\newcommand{\Th}{\Theta}
\newcommand{\La}{\Lambda}
\newcommand{\Si}{\Sigma}
\newcommand{\Om}{\Omega}

% % Mathbb
\newcommand{\A}{\mathbb{A}}
\newcommand{\I}{\mathbb{I}}
\newcommand{\N}{\mathbb{N}}
\newcommand{\M}{\mathbb{M}}
\newcommand{\Z}{\mathbb{Z}}
\newcommand{\Q}{\mathbb{Q}}
\newcommand{\R}{\mathbb{R}}
\newcommand{\C}{\mathbb{C}}
\newcommand{\F}{\mathbb{F}}
\newcommand{\V}{\mathbb{V}}
\newcommand{\U}{\mathbb{U}}

% % Mathcal
\renewcommand{\AA}{\mathcal{A}}
\newcommand{\BB}{\mathcal{B}}
\newcommand{\CC}{\mathcal{C}}
\newcommand{\DD}{\mathcal{D}}
\newcommand{\EE}{\mathcal{E}}
\newcommand{\FF}{\mathcal{F}}
\newcommand{\GG}{\mathcal{G}}
\newcommand{\HH}{\mathcal{H}}
\newcommand{\II}{\mathcal{I}}
\newcommand{\JJ}{\mathcal{J}}
\newcommand{\KK}{\mathcal{K}}
\newcommand{\LL}{\mathcal{L}}
\newcommand{\MM}{\mathcal{M}}
\newcommand{\NN}{\mathcal{N}}
\newcommand{\OO}{\mathcal{O}}
\newcommand{\PP}{\mathcal{P}}
\newcommand{\QQ}{\mathcal{Q}}
\newcommand{\RR}{\mathcal{R}}
\renewcommand{\SS}{\mathcal{S}}
\newcommand{\TT}{\mathcal{T}}
\newcommand{\UU}{\mathcal{U}}
\newcommand{\VV}{\mathcal{V}}
\newcommand{\WW}{\mathcal{W}}
\newcommand{\XX}{\mathcal{X}}
\newcommand{\YY}{\mathcal{Y}}
\newcommand{\ZZ}{\mathcal{Z}}

% % Mathfrak
\newcommand{\f}[1]{\mathfrak{#1}}

% % Mathrsfs
\newcommand{\s}[1]{\mathscr{#1}}

% % Category Theory
\newcommand{\Obj}[1]{\mathrm{Obj}\left(#1\right)}
\newcommand{\End}[2]{\mathrm{End}_{#1}#2\,}
\newcommand{\Aut}[2]{\mathrm{Aut}_{#1}#2\,}
\newcommand{\op}{^{\textsf{op}}}
\newcommand{\iso}{\cong}
\newcommand{\id}[1]{\mathbbm{1}_{#1}}
\newcommand{\hookr}{\hookrightarrow}
\newcommand{\yo}{\mathbf{y}}
\newcommand{\TERM}{\mathbf{1}}
\DeclareMathOperator{\colim}{colim}
\DeclareMathOperator{\dom}{dom}
\DeclareMathOperator{\codom}{codom}

\newcommand{\CAT}{\mathbf{Cat}}
\newcommand{\SET}{\mathbf{Set}}
\newcommand{\PSH}[1]{\mathbf{PSh}\brkt{#1}}
\newcommand{\SH}[1]{\mathbf{Sh}\brkt{#1}}
% \newcommand{\TOP}{\mathbf{Top}}
%\newcommand{\GRP}{\mathbf{Grp}}
% \newcommand{\RING}{\mathbf{Ring}}
% \newcommand{\MOD}[1][R]{#1\text{-}\mathbf{Mod}}
% \newcommand{\VEC}[1][K]{#1\text{-}\mathbf{Vec}}
% \newcommand{\ALG}[1][R]{#1\text{-}\mathbf{Alg}}



% % xcolor
\definecolor{darkgrey}{gray}{0.10}
\definecolor{lightgrey}{gray}{0.50}
\definecolor{slightgrey}{gray}{0.80}
\definecolor{softblue}{RGB}{30,100,200}
\definecolor{syntax}{RGB}{52, 72, 115}
\definecolor{semantics}{RGB}{168, 50, 82}

% % hyperref
\hypersetup{
      colorlinks = true,
      linkcolor = {softblue},
      citecolor = {blue}
}

\newcommand{\link}[1]{\hypertarget{#1}{}}
\newcommand{\linkto}[2]{\hyperlink{#1}{#2}}

% Theorems

% % custom theoremstyles
\newtheoremstyle{definitionstyle}
{0pt}% above thm
{0pt}% below thm
{}% body font
{}% space to indent
{\bf}% head font
{\vspace{1mm}}% punctuation between head and body
{\newline}% space after head
{\thmname{#1}\thmnote{\,\,--\,\,#3}}

% % custom theoremstyles
\newtheoremstyle{propositionstyle}
{0pt}% above thm
{0pt}% below thm
{}% body font
{}% space to indent
{\bf}% head font
{\vspace{1mm}}% punctuation between head and body
{\newline}% space after head
{\thmname{#1}\thmnote{\,\,--\,\,#3}}

\newtheoremstyle{exercisestyle}%
{0pt}% above thm
{0pt}% below thm
{\it}% body font
{}% space to indent
{\scshape}% head font
{.}% punctuation between head and body
{ }% space after head
{\thmname{#1}\thmnote{ (#3)}}

\newtheoremstyle{remarkstyle}%
{0pt}% above thm
{0pt}% below thm
{}% body font
{}% space to indent
{\it}% head font
{.}% punctuation between head and body
{ }% space after head
{\thmname{#1}\thmnote{\,\,--\,\,#3}}

% % Theorem environments

\theoremstyle{definitionstyle}
\newmdtheoremenv[
    %skipabove = \baselineskip
    linewidth = 1.2pt,
    leftmargin = 5pt,
    rightmargin = 0pt,
    linecolor = darkgrey,
    topline = false,
    bottomline = false,
    rightline = false,
    footnoteinside = true
]{dfn}{Definition}
\newmdtheoremenv[
    linewidth = 1 pt,
    leftmargin = 5pt,
    rightmargin = 0pt,
    linecolor = lightgrey,
    topline = false,
    bottomline = false,
    rightline = false,
    footnoteinside = true
]{prop}{Proposition}
\newmdtheoremenv[
    linewidth = 1 pt,
    leftmargin = 5pt,
    rightmargin = 0pt,
    linecolor = lightgrey,
    topline = false,
    bottomline = false,
    rightline = false,
    footnoteinside = true
]{cor}{Corollary}
\newmdtheoremenv[
    linewidth = 1 pt,
    leftmargin = 5pt,
    rightmargin = 0pt,
    linecolor = lightgrey,
    topline = false,
    bottomline = false,
    rightline = false,
    footnoteinside = true
]{lem}{Lemma}
\newmdtheoremenv[
    linewidth = 1.2pt,
    leftmargin = 5pt,
    rightmargin = 0pt,
    linecolor = darkgrey,
    topline = false,
    bottomline = false,
    rightline = false,
    footnoteinside = true
]{thm}{Theorem}

\theoremstyle{exercisestyle}
\newmdtheoremenv[
    linewidth = 0.7 pt,
    leftmargin = 20pt,
    rightmargin = 0pt,
    linecolor = darkgrey,
    topline = false,
    bottomline = false,
    rightline = false,
    footnoteinside = true
]{ex}{Exercise}
\newmdtheoremenv[
    linewidth = 0.7 pt,
    leftmargin = 20pt,
    rightmargin = 0pt,
    linecolor = darkgrey,
    topline = false,
    bottomline = false,
    rightline = false,
    footnoteinside = true
]{eg}{Example}
\newmdtheoremenv[
    linewidth = 0.7 pt,
    leftmargin = 20pt,
    rightmargin = 0pt,
    linecolor = darkgrey,
    topline = false,
    bottomline = false,
    rightline = false,
    footnoteinside = true
]{nttn}{Notation}

\theoremstyle{remarkstyle}
\newtheorem{rmk}{Remark}

% % footnotes
\renewcommand{\thempfootnote}{$\dagger$}
\MakePerPage{footnote}

% % tikzcd diagram
\newenvironment{cd}{
    \begin{figure}[H]
    \centering
    \begin{tikzcd}
}{
    \end{tikzcd}
    \end{figure}
}

% % Section titles
\titlespacing*{\section}
{0pt}{5ex plus 1ex minus .2ex}{4ex plus .2ex}
\titlespacing*{\subsection}
{0pt}{5ex plus 1ex minus .2ex}{4ex plus .2ex}

% Syntax highlighting for lean
\usepackage{color}
\definecolor{keywordcolor}{rgb}{0.6, 0.3, 0.6}   % light purple
\definecolor{tacticcolor}{rgb}{0.2, 0.2, 0.6}    % dark purple
\definecolor{commentcolor}{rgb}{0.4, 0.4, 0.4}   % grey
\definecolor{symbolcolor}{rgb}{0.7, 0.2, 0.2}    % blue
\definecolor{sortcolor}{rgb}{0.2, 0.7, 0.7}      % cyan
\definecolor{attributecolor}{RGB}{120,0,0}       % maroon
\definecolor{draculaGrey}{RGB}{25, 27, 15}       % grey

\def\lstlanguagefiles{lstlean.tex}

% set default language
\lstset{language=lean}

\begin{document}
\title{Equational Logical Frameworks as Locally Cartesian Closed Categories}
\author{Joseph Hua}
\maketitle

\tableofcontents

\section{Overview}

\begin{itemize}
  \item Introduce LF with equality. Closely follow \cite{harper2021equational}
  \item Introduce the categorical way of looking at the type formers (motivational)
  \item Construct category of contexts formally
  \item General model of LF in an LCCC
  \item Lawvere style correspondence
\end{itemize}

In the first section, we present an \textit{equational logical framework} as in \cite{harper2021equational},
which can be used to define dependent type theories such as Martin L{\"o}f type theory or
\textsf{ModTT} \cite{Sterling_2021}.

\subsection{Notation}

We follow \cite{harper2021equational} and use the word ``class'' for what is usually called ``type'' and
the word ``object'' for what is usually called ``term''.
This is to distinguish between the meta-level class in the logical framework
and types internal to the type theory defined in the logical framework.
To not confuse this with the categorical notion of object,
we will write $\CC$-object for an object in a category $\CC$.

\section{An Equational Logical Framework}
\link{sec_2}

\begin{figure}[h]
\begin{mathpar}
  \infer[\textsc{ctx-emp}]{\CtxJdg{\bullet}}{}

  \infer[\textsc{ctx-ext}]{\CtxJdg{\CtxExt{\Ga}{X}{K}}}{\ClsJdg{\Ga}{K}}

  \infer[\textsc{ctx-hd}]{X:K \in{\CtxExt{\Ga}{X}{K}}}
  {}

  \infer[\textsc{ctx-tl}]{X_0:K_0 \in \CtxExt{\Ga}{X_1}{K_1}}
    {X_0:K_0 \in \Ga}
\end{mathpar}
  \caption{Context formation}
  \label{ctx_form}
\end{figure}

\begin{figure}[h]
\begin{mathpar}
  \infer[\textsc{sort-cls}]{\ClsJdg{\Ga}{\SortCS}}{\CtxJdg{\Ga}}

  \infer[\textsc{incl-cls}]{\ClsJdg{\Ga}{S}}{\ObjJdg{\Ga}{S}{\SortCS}}

  \infer[\textsc{pi-cls}]{\ClsJdg{\Ga}{\PiCS{X}{S}{K}}}
  {\ObjJdg{\Ga}{S}{\SortCS} & \ClsJdg{\CtxExt{\Ga}{X}{S}}{K}}

  % \infer[\textsc{eq-cls}]{\ClsJdg{\Ga}{\EqCS{S}{O_0}{O_1}}}
  %   {\ObjJdg{\Ga}{O_0}{K} & \ObjJdg{\Ga}{O_1}{K}}
\end{mathpar}
  \caption{Class formation}
  \label{cls_form}
\end{figure}

\begin{figure}[h]
\begin{mathpar}
  \infer[\textsc{pi-sort}]{\ObjJdg{\Ga}{\PiCS{X}{S_0}{S_1}}{\SortCS}}
    {\ObjJdg{\Ga}{S_0}{\SortCS} & \ObjJdg{\CtxExt{\Ga}{X}{S_0}}{S_1}{\SortCS}}

  \infer[\textsc{eq-sort}]{\ObjJdg{\Ga}{\EqCS{S}{O_0}{O_1}}{\SortCS}}
    {\ObjJdg{\Ga}{S}{\SortCS} & \ObjJdg{\Ga}{O_0}{S} & \ObjJdg{\Ga}{O_1}{S}}
\end{mathpar} \caption{Sort formation}
  \label{sort_form}
\end{figure}

\begin{figure}[h]
\begin{mathpar}
  \infer[\textsc{var-obj}]{\ObjJdg{\Ga}{X}{K}}
    {\CtxJdg{\Ga} & X:K \in \Ga}

  \infer[\textsc{pi-lam-obj}]{\ObjJdg{\Ga}{\LamCS{X}{S}{O}}{\PiCS{X}{S}{K}}}
    {\ObjJdg{\Ga}{S}{\SortCS} & \ObjJdg{\CtxExt{\Ga}{X}{S}}{O}{K}}

  \infer[\textsc{pi-app-obj}]{\ObjJdg{\Ga}{\ApCS{O_0}{O_1}}{\Subst{O_1}{X}{K}}}
    {\ObjJdg{\Ga}{O_0}{\PiCS{X}{S}{K}} & \ObjJdg{\Ga}{O_1}{S}}

  \infer[\textsc{eq-self-obj}]{\ObjJdg{\Ga}{\SelfCS}{\EqCS{K}{O}{O}}}
    {\ObjJdg{\Ga}{O}{K}}

  \infer[\textsc{je-obj}]{\ObjJdg{\Ga}{O}{K_1}}
    {\ObjJdg{\Ga}{O}{K_0} & \IdClsJdg{\Ga}{K_0}{K_1}}
\end{mathpar}
  \caption{Object formation}
  \label{obj_form}
\end{figure}

\begin{figure}
  \begin{mathpar}
  \infer[\textsc{cls-rfl}]{\IdClsJdg{\Ga}{K}{K}}{\ClsJdg{\Ga}{K}}

  \infer[\textsc{cls-st}]{\IdClsJdg{\Ga}{K_0}{K_1}}
    {\IdClsJdg{\Ga}{K_0}{K_2} && \IdClsJdg{\Ga}{K_1}{K_2}}

  \infer[\textsc{obj-rfl}]{\IdObjJdg{\Ga}{O}{O}{K}}{\ObjJdg{\Ga}{O}{K}}

  \infer[\textsc{obj-st}]{\IdObjJdg{\Ga}{O_0}{O_1}{K}}
    {\IdObjJdg{\Ga}{O_0}{O_2}{K} && \IdObjJdg{\Ga}{O_1}{O_2}{K}}

  \infer[\textsc{obj-je-cls}]{\IdObjJdg{\Ga}{O_0}{O_1}{K_1}}{\IdObjJdg{\Ga}{O_0}{O_1}{K_0} && \IdClsJdg{\Ga}{K_0}{K_1}}
      \end{mathpar}

      \begin{mathpar}
  \infer[\textsc{app-lam}]{\IdObjJdg{\Ga}{(\LamCS{X}{S}{O_0}) O_1}{\Subst{O_1}{X}{O_0}}
      {\Subst{O_1}{X}{K}}}{\ObjJdg{\CtxExt{\Ga}{X}{S}}{O_0}{K} && \ObjJdg{\Ga}{O_1}{S}}

    \infer[\textsc{lam-app}]{\IdObjJdg{\Ga}{O}{\LamCS{X}{S}{(O X)}}{\PiCS{X}{S}{K}}}{\ObjJdg{\Ga}{O}{\PiCS{X}{S}{K}}}

    \infer[\textsc{reflection}]{\IdObjJdg{\Ga}{O_1}{O_2}{S}}{\ObjJdg{\Ga}{O_0}{\EqCS{S}{O_1}{O_2}}}

    \infer[\textsc{unicity}]{\IdObjJdg{\Ga}{O_0}{O_3}{\EqCS{S}{O_1}{O_2}}}
      {\ObjJdg{\Ga}{O_0}{\EqCS{S}{O_1}{O_2}} && \ObjJdg{\Ga}{O_3}{\EqCS{S}{O_1}{O_2}}}

      \end{mathpar}

      \begin{mathpar}
  \infer[\textsc{je-incl-cls}]{\IdClsJdg{\Ga}{S_0}{S_1}}{\IdObjJdg{\Ga}{S_0}{S_1}{\SortCS}}

  \infer[\textsc{je-pi-cls}]{\IdClsJdg{\Ga}{\PiCS{X}{S_0}{K_0}}{\PiCS{X}{S_1}{K_1}}}
  {\IdObjJdg{\Ga}{S_0}{S_1}{\SortCS} & \IdClsJdg{\CtxExt{\Ga}{X}{S_0}}{K_0}{K_1}}

    \infer[\textsc{je-pi-sort}]{\IdObjJdg{\Ga}{\PiCS{X}{S_0}{S_2}}{\PiCS{X}{S_1}{S_3}}{\SortCS}}
    {\IdObjJdg{\Ga}{S_0}{S_1}{\SortCS} & \IdObjJdg{\CtxExt{\Ga}{X}{S_0}}{S_2}{S_3}{{\SortCS}}}

  \infer[\textsc{je-eq-sort}]{\IdObjJdg{\Ga}{\EqCS{S_0}{O_0}{O_2}}{\EqCS{S_0}{O_1}{O_3}}{\SortCS}}
    {\IdObjJdg{\Ga}{S_0}{S_1}{\SortCS} & \IdObjJdg{\Ga}{O_0}{O_1}{S_0} & \IdObjJdg{\Ga}{O_2}{O_3}{S_1}}

  \infer[\textsc{je-pi-lam-obj}]{\IdObjJdg{\Ga}{\LamCS{X}{S_0}{O}}{\LamCS{X}{S_1}{O_1}}{\PiCS{X}{S_0}{K}}}
    {\IdObjJdg{\Ga}{S_0}{S_1}{\SortCS} & \IdObjJdg{\CtxExt{\Ga}{X}{S_0}}{O_0}{O_1}{K}}

  \infer[\textsc{je-pi-app-obj}]{\IdObjJdg{\Ga}{\ApCS{O_0}{O_2}}{\ApCS{O_1}{O_3}}{\Subst{O_2}{X}{K}}}
    {\IdObjJdg{\Ga}{O_0}{O_1}{\PiCS{X}{S}{K}} & \IdObjJdg{\Ga}{O_2}{O_3}{S}}
  \end{mathpar}
  \caption{Judgmental equality}
  \label{id_form}
\end{figure}

In this section we recall the equational logical framework,
as described in \cite{harper2021equational} (with some minor changes).
This LF can be used to define a wide variety of type theories,
including dependent type theories such as Martin L{\"o}f type theory or
\textsf{ModTT} \cite{Sterling_2021}.
Within the LF we define type theories by asserting the rules of the type theory
via a collection of generators, called a signature and denoted $\Si$.
We state only state the syntax and judgment rules
and refer to \cite{harper2021equational} for details and examples.

\begin{align*}
  &\textit{variable} \quad & X {\color{white} ::=} \quad & \\
  &\textit{expression} \quad & O, S, K, e ::= \quad & X \BNFSep \SortCS \BNFSep \PiCS{X}{S}{e}
                                     \BNFSep \EqCS{S}{O_{0}}{O_{1}} \BNFSep \SelfCS \BNFSep \LamCS{X}{S}{O} \BNFSep \ApCS{O_{0}}{O_{1}} \\
  &\textit{context} \quad & \Ga, \De ::= \quad & \cdot \BNFSep \CtxExt{\Ga}{X}{K}
\end{align*}
$O$ indicates expressions representing objects
(though just an expression at this point),
similarly $S$ represents sorts and $K$ represents classes.
The agnostic expression symbol $e$ represents sorts or a classes.

The judgments we define are
\begin{align*}
  &\CtxJdg{\Ga} && \Ga \text{ is a context }  \\
  &X:K \in \Ga && X:K \text{ appears in context } \Ga \\
  &\ClsJdg{\Ga}{K} && K \text{ is a class in context } \Ga \\
  &\ObjJdg{\Ga}{O}{K} && O \text{ is an object of class } K
    \text{ in context } \Ga && \\
  &\IdClsJdg{\Ga}{K_{0}}{K_{1}} && K_{0}, K_{1}
    \text{ are equal classes in context } \Ga \\
  &\IdObjJdg{\Ga}{O_{0}}{O_{1}}{K} && O_{0},O_{1}
    \text{ are equal objects of class } K
    \text{ in context } \Ga
\end{align*}

The legal judgments given by the closure under the judgment rules
in figures \ref{ctx_form}, \ref{cls_form}, \ref{sort_form}, \ref{obj_form},
\ref{id_form}.

Note that unlike in \cite{harper2021equational} we have \textsc{eq-sort}
creating a sort rather than a class.
This will ensure one of our display categories is an LCCC.

Judgmental equality for classes $\IdClsJdg{\Ga}{K_{0}}{K_{1}}$ and objects
$\IdObjJdg{\Ga}{O_{0}}{O_{1}}{K}$ are defined to be congruences,
i.e. equivalence relations that respect formation of classes,
sorts, and objects.
Furthermore, we have $\be$ and $\eta$ rules for \textsc{pi}.

Finally, signatures are defined as contexts $\CtxJdg{\Si}$,
and the type theory generated by a signature $\Si$ consists of
legal judgments over $\Si$.
Note that before introducing a signature we cannot make any sorts,
nor can we make any interesting objects.
The only rules for their formation are via \textsc{pi-sort} and \textsc{eq-sort},
which require more sorts and objects as premises.
It is the signature that populates \textsf{Sort} with sorts
and populates those sorts with objects.
Given a signature $\Si$, we write $\LF{\Si}$ to denote the
generated type theory.

\begin{align*}
  &\SiCtxJdg{\Ga} & :=
  & & & \CtxJdg{\Si,\Ga} \\
  &\SiClsJdg{\Ga}{K} & :=
  & & & \ClsJdg{\Si,\Ga}{K} \\
  &\SiObjJdg{\Ga}{O}{K} & :=
  & & & \ObjJdg{\Si,\Ga}{O}{K} \\
  &\SiIdClsJdg{\Ga}{K_{0}}{K_{1}} & :=
  & & & \IdClsJdg{\Si,\Ga}{K_{0}}{K_{1}} \\
  &\SiIdObjJdg{\Ga}{O_{0}}{O_{1}}{K} & :=
  & & & \IdObjJdg{\Si,\Ga}{O_{0}}{O_{1}}{K}
\end{align*}


\section{Categorical definitions}

\begin{dfn}[Cartesian closed category (CCC)]
  \link{dfn_CCC}
  We say a category $\CC$ is cartesian closed when
  \begin{itemize}
    \item $\CC$ has finite products
    \item For each $\CC$-object $A$, the product with $A$ functor $\times A : \CC \to \CC$ has a right adjoint.
  \end{itemize}
  We denote the right adjoint by $[A,-]$ and call it internal hom.
\end{dfn}

\begin{dfn}[Locally cartesian closed category (LCCC)]
  We say a category $\CC$ is locally cartesian closed when either of the equivalent definitions hold
  \begin{enumerate}
    \item Every slice category $\CC / A$ over a $\CC$-object $A$ is cartesian closed.
    \item $\CC$ has pullbacks, and for each morphism $f : B \to A$ in $\CC$,
          the base change (pullback along $f$) $f^{*} : \CC / A \to \CC / B$ has a right adjoint.
  \end{enumerate}
  We denote the right adjoint to base change by $\Pi_{f}$.
\end{dfn}

We do not show that these two definitions are equivalent.
The idea is that pullbacks correspond to products in slices
and $\Pi$s correspond to internal homs in slices.

\section{Intuition for considering LCCCs}

We want to find categorical semantics for our type theory.
This amounts to one of two equivalent things
\begin{itemize}
  \item View each aspect of the type theory in a category $\CC$
  \item Construct a syntactic category (which will be the ``category of contexts'')
        and take semantics to be anything that imitates it (in a Platonic form sense).
\end{itemize}

For this section we refer to $\STLC$,
a simply typed $\la$-calculus with only unit, product and function classes.
And use intuition of its categorical semantics to motivate those for logical frameworks.

Recall $\STLC$ can be interpreted in a \linkto{dfn_CCC}{CCC} $\CC$ by
taking classes $K$ as $\CC$-objects $\bbrkt{K}$,
contexts $\CtxExt{X_{1}:K_{1},\dots}{X_{n}}{K_{n}}$
as products $\bbrkt{K_{1}} \tdt \bbrkt{K_{n}}$,
and objects $\ObjJdg{\Ga}{K}{O}$ as morphisms $\bbrkt{O} : \bbrkt{\Ga} \to \bbrkt{K}$. % % Ref needed
This seems quite natural, since product classes become products,
and function classes become internal homs in the semantics.
If we are to immitate this for $\LF{\Si}$,
we would only be able to talk about \textit{closed classes} $\ClsJdg{\bullet}{K}$,
since in $\STLC$ we don't have dependent classes $\ClsJdg{\Ga}{K}$.
However, in the presence of dependent pairs (aka $\Si$-types) we
could view a context $\Ga = \CtxExt{X_{1}:K_{1},\dots}{X_{n}}{K_{n}}$
as a closed class $\Si_{\Ga} := \Si_{X_{1}:K_{1}} \cdots \Si_{X_{n-1}:K_{n-1}} K_{n}$.
Then any dependent class can be viewed as a context, which can be viewed as a closed class.

\subsection{Contexts and dependent classes}

Instead of including dependent pairs,
let us view dependent classes $\ClsJdg{\Ga}{K}$ as extended contexts
$\CtxExt{\Ga}{X}{K}$, and first interpret \textit{contexts} as $\CC$-objects,
and \textit{as a consequence} dependent classes as $\CC$-objects.

\begin{multicols}{2}
  \color{syntax}{
  \[\CtxJdg{\Ga}\]}

  \color{semantics}{
  \[\bbrkt{\Ga} \in \CC\]
  }
\end{multicols}

It is then natural to define morphisms of contexts (aka a substitution),
denoted $\ObjJdg{\Ga}{\si}{\De}$,
consisting of the data of objects

\begin{align*}
  &\ObjJdg{\Ga}{\si_{1}}{K_{1}}\\
  &\ObjJdg{\Ga}{\si_{2}}{\Subst{\si_{1}}{X_{1}}{K_{2}}}\\
  &\vdots & \text{ when } \De = \CtxExt{X_{1}:K_{1}}{\cdots, X_{n}}{K_{n}}\\
  &\ObjJdg{\Ga}{\si_{n}}{\Subst{\si_{n-1}}{X_{n-1}}{\cdots \Subst{\si_{1}}{X_{1}}{K_{n}}}}
\end{align*}

Such a substitution corresponds to a closed object of the closed class
$\Si_{\Ga} \to \Si_{\De}$ in the presence of dependent pairs.

\begin{multicols}{2}
  \color{syntax}{
  \[\ObjJdg{\Ga}{\si}{\De}\]}

  \color{semantics}{
  \[\bbrkt{\si} : \bbrkt{\Ga} \to \bbrkt{\De}\]
  }
\end{multicols}

The empty context gives an object $\bbrkt{\bullet}$,
and for any context $\Ga$ there is exactly one substitution $\ObjJdg{\Ga}{\si}{\bullet}$,
given by the data of no objects.
This makes $\bbrkt{\bullet}$ the terminal object.

\begin{multicols}{2}
  \color{syntax}{
  \[\infer[\textsc{ctx-emp}]{\CtxJdg{\bullet}}{}\]}

  \color{semantics}{
  \[\TERM_{\CC} \in \CC\]
  }
\end{multicols}

For context extension, given $\ClsJdg{\Ga}{K}$,
we obtain a substitution $\ObjJdg{\CtxExt{\Ga}{X}{K}}{\si}{\Ga}$
by $\ObjJdg{\CtxExt{\Ga}{X}{K}}{X_{i}}{K_{i}}$ for each $X_{i} : K_{i} \in \Ga$.
Then let's take the interpretation of $K$ to be a $\CC/\bbrkt{\Ga}$-object,
where $\CC/\bbrkt{\Ga}$ is the slice category.

\[ \bbrkt{\ClsJdg{\Ga}{K}} := \bbrkt{\CtxExt{\Ga}{X}{K}} \text{ and } \bbrkt{\ObjJdg{\CtxExt{\Ga}{X}{K}}{\si}{\Ga}}\]

We can informally identify use the slice-object $\bbrkt{\ClsJdg{\Ga}{K}}$ to denote the underlying morphism.

\begin{multicols}{2}
  \color{syntax}{
  \[\infer[\textsc{ctx-ext}]{\CtxJdg{\CtxExt{\Ga}{X}{K}}}{\ClsJdg{\Ga}{K}}\]}


  \color{semantics}{
    \begin{cd}
     & \\
  	{\bbrkt{\CtxExt{\Ga}{X}{K}}} && {\bbrkt{\Ga}}
	\arrow["{\bbrkt{\ClsJdg{\Ga}{K}}}", from=2-1, to=2-3]
  \end{cd}
  }
\end{multicols}

\subsection{Objects}

In the $\STLC$ case we interpreted objects
$\ObjJdg{\Ga}{K}{O}$ as morphisms $\bbrkt{O} : \bbrkt{\Ga} \to \bbrkt{K}$ in $\CC$.
Since the classes are now dependent,
and the object is situated in some context,
we should upgrade this to a morphism in the slice $\CC / \bbrkt{\Ga}$.

\begin{multicols}{2}

    \[\color{syntax}\ObjJdg{\Ga}{O}{K}\]


  \color{semantics}{
  \begin{cd}
     {\bbrkt{\Ga}} && {\bbrkt{\Ga,X:K}} \\
    & {\bbrkt{\Ga}}
    \arrow["{\bbrkt{\ObjJdg{\Ga}{O}{K}}}", from=1-1, to=1-3]
    \arrow["{\id{\bbrkt{\Ga}}}"', from=1-1, to=2-2]
    \arrow["{\bbrkt{\ClsJdg{\Ga}{K}}}", from=1-3, to=2-2]
  \end{cd}
  }
\end{multicols}

We can call $\bbrkt{\ObjJdg{\Ga}{O}{K}}$
a section of the bundle $\bbrkt{\ClsJdg{\Ga}{K}}$
in the sense that for any collection of objects $O_{1},\cdots,O_{n}$ in the classes from $\Ga$,
there is a closed class \[\ClsJdg{\bullet}{[O_{n}/X_{n}] \cdots \Subst{O_{1}}{X_{1}}{K}}\]
A section of the bundle then picks out for any such collection of objects,
an object \[\ObjJdg{\bullet}{[O_{n}/X_{n}] \cdots \Subst{O_{1}}{X_{1}}{O}}{[O_{n}/X_{n}] \cdots \Subst{O_{1}}{X_{1}}{K}}\]
\begin{multicols}{2}

    \[\color{syntax}  \infer[\textsc{var-obj}]{\ObjJdg{\Ga}{X}{K}}
    {\CtxJdg{\Ga} & X:K \in \Ga}\]


  \color{semantics}{
  \begin{cd}
     {\bbrkt{\Ga}} && {\bbrkt{\Ga,Y:K}} \\
    & {\bbrkt{\Ga}}
    \arrow["{\bbrkt{\ObjJdg{\Ga}{X}{K}}}", from=1-1, to=1-3]
    \arrow["{\id{\bbrkt{\Ga}}}"', from=1-1, to=2-2]
    \arrow["{\bbrkt{\ClsJdg{\Ga}{K}}}", from=1-3, to=2-2]
  \end{cd}
  }
\end{multicols}

We may also write $\bbrkt{K}$ to mean $\bbrkt{\ClsJdg{\Ga}{K}}$ for brevity.

\subsection{Substitution}

Context morphisms (aka substitutions) are dependent tuples of objects,
just as contexts are dependent tuples of classes.
They can be {\color{semantics} composed}, which we can view as {\color{syntax} sequential substitution}.
They can induce {\color{syntax} substitutions on dependent classes},
which we will show corresponds to {\color{semantics} pullback of bundles}.
They can induce {\color{syntax} substitution on objects} (more generally {\color{syntax} substitution on substitutions}),
which we will show corresponds to {\color{semantics} pullback of sections}
(more generally {\color{semantics} existence of all pullbacks}).

First let us suppose $\ObjJdg{\Ga}{\si}{\De}$ is a substitution and $\ClsJdg{\De}{K}$.
Then we can form $\ClsJdg{\Ga}{\si K}$,
where
\[ \si K := \Subst{\si_{n}}{X_{n}}{ \cdots \Subst{\si_{1}}{X_{1}}{K}} \quad \text{ and }
  \quad \De = \CtxExt{X_{1}:K_{1}}{\cdots, X_{n}}{K_{n}}\]

This automatically gives us a substitution $\ObjJdg{\CtxExt{\Ga}{X}{\si K}}{\si :: X}{{\CtxExt{\De}{X}{K}}}$.
This says we have a pullback

\begin{multicols}{2}

\[\color{syntax} \infer{\ClsJdg{\Ga}{\si K}}{\ObjJdg{\Ga}{\si}{\De} & & \ClsJdg{\De}{K} } \]

%% this admissible judgment needs a name

{\color{semantics}
\begin{cd}
  {\bbrkt{\CtxExt{\Ga}{X}{\si K}}} && {\bbrkt{\CtxExt{\De}{X}{K}}} \\
  \\
  {\bbrkt{\Ga}} && {\bbrkt{\De}}
  \arrow["{\bbrkt{\si}}"', from=3-1, to=3-3]
  \arrow["{\bbrkt{K}}", from=1-3, to=3-3]
  \arrow["{\bbrkt{\si :: X}}", dashed, from=1-1, to=1-3]
  \arrow["{\bbrkt{\si K}}"', dashed, from=1-1, to=3-1]
  \arrow["\lrcorner"{anchor=center, pos=0.125}, draw=none, from=1-1, to=3-3]
\end{cd}}

\end{multicols}

To check the diagram commutes, we use a notion of judgmental equality between substitutions,
which we can define component-wise by judgmental equality between objects.
For the universal property we need to apply (syntactic) substitution on objects.

Now suppose $\ObjJdg{\De}{O}{K}$.
Then we can form $\ObjJdg{\Ga}{\si O}{\si K}$,
where
\[ \si O := \Subst{\si_{n}}{X_{n}}{ \cdots \Subst{\si_{1}}{X_{1}}{O}}\]

This says our pullback diagram extends

\begin{multicols}{2}

\[\color{syntax} \infer{\ObjJdg{\Ga}{\si O}{\si K}}{\ObjJdg{\Ga}{\si}{\De} & & \ObjJdg{\De}{O}{K} } \]

%% this admissible judgment needs a name

{\color{semantics}
\begin{cd}
  	{\bbrkt{\Ga}} && {\bbrkt{\De}} \\
	\\
	{\bbrkt{\CtxExt{\Ga}{X}{\si K}}} && {\bbrkt{\CtxExt{\De}{X}{K}}} \\
	\\
	{\bbrkt{\Ga}} && {\bbrkt{\De}}
	\arrow["{\bbrkt{\si}}"', from=5-1, to=5-3]
	\arrow["{\bbrkt{K}}", from=3-3, to=5-3]
	\arrow["{\bbrkt{\si :: X}}", from=3-1, to=3-3]
	\arrow["{\bbrkt{\si K}}"', from=3-1, to=5-1]
	\arrow["\lrcorner"{anchor=center, pos=0.125}, draw=none, from=3-1, to=5-3]
	\arrow["{\bbrkt{\si O}}"', dashed, from=1-1, to=3-1]
	\arrow["{\bbrkt{O}}", from=1-3, to=3-3]
	\arrow["{\bbrkt{\si}}", from=1-1, to=1-3]
	\arrow["\lrcorner"{anchor=center, pos=0.125}, draw=none, from=1-1, to=3-3]
\end{cd}}
\end{multicols}

Now we can use substitution on classes and objects to define composition of context morphisms.
Given two substitutions $\ObjJdg{\Ga_{0}}{\si}{\Ga_{1}}$ and $\ObjJdg{\Ga_{1}}{\rho}{\Ga_{2}}$
we define the composition $\ObjJdg{\Ga_{0}}{\si \gg \rho}{\Ga_{2}}$
by giving

\begin{align*}
  &\ObjJdg{\Ga_{0}}{\si{\rho_{1}}}{\si K_{1}}\\
  &\ObjJdg{\Ga_{0}}{\si\rho_{2}}{\si \Subst{\rho_{1}}{X_{1}}{K_{2}}} \\
  &\vdots & \text{ where } & \Ga_{2} = \CtxExt{X_{1}:K_{1}}{\cdots, X_{n}}{K_{n}}\\
  &\ObjJdg{\Ga_{0}}{\si \rho_{n}}{\si \Subst{\rho_{n-1}}{X_{n-1}}{\cdots \Subst{\rho_{1}}{X_{1}}{K_{n}}}}
\end{align*}

So we have

\begin{multicols}{2}
  \[\color{syntax} \ObjJdg{\Ga_{0}}{\si \gg \rho}{\Ga_{2}}\]

   \[ \color{semantics}
    \bbrkt{\rho} \circ \bbrkt{\si} : \bbrkt{\Ga_0} \to \bbrkt{\Ga_2} \]
\end{multicols}

There is a techincal issue here about substitution not working out in a general LCCC,
because of pullback only being unique up to isomorphism.
We delay this issue until later,
and assume strict pullbacks,
i.e. all pullbacks in $\CC$ of the same diagram are equal. %% reference + link to later

\subsection{Equality classes}
In the previous subsection we showed that the semantics category $\CC$
pullbacks of context morphisms along context extensions/classes.
By (assume strict) pullback pasting \cite{pullback_pasting} we even have pullback of context morphisms along
a finite composition of context extensions.
Do we have pullback along any substitution?
The answer is yes, given the presence of an equality class.

% https://q.uiver.app/?q=WzAsNCxbMCwwLCJcXERlIFxcdGltZXNfe1xcR2F9IEUiXSxbMiwwLCJFIl0sWzAsMiwiXFxEZSJdLFsyLDIsIlxcR2EiXSxbMCwxXSxbMiwzLCJcXGRlIiwyXSxbMSwzLCJlIl0sWzAsMl1d
\[\begin{tikzcd}
  {\bbrkt{\De} \times_{\bbrkt{\Ga}} \bbrkt{E}} && \bbrkt{E} \\
  \\
  \bbrkt{\De} && \bbrkt{\Ga}
  \arrow["\pi_{E}", from=1-1, to=1-3]
  \arrow["\bbrkt{\de}"', from=3-1, to=3-3]
  \arrow["\bbrkt{\ep}", from=1-3, to=3-3]
  \arrow["\pi_{\De}"', from=1-1, to=3-1]
  \arrow["\lrcorner"{anchor=center, pos=0.125}, draw=none, from=1-1, to=3-3]
\end{tikzcd}\]

Again imagining $\De$ and $E$ as closed classes,
the pullback should be ``pairs of objects in $\De \times E$'' that
are equal upon their substitutions into the objects given by $\de$ and $\ep$.
In set-theoretic notation (since we are inspired by pullback in $\SET$)
\[ \set{ (X,Y) \in \De \times E \st \de(X) = \ep(Y) }\]
Translating this idea into a context means that we need

\begin{align*}
  {\bbrkt{\De} \times_{\bbrkt{\Ga}} \bbrkt{E}} & = \bbrkt{X : \De, Y : E, H : \EqCS{\Ga}{\de}{\ep}}\\
  \pi_{\De} & = \bbrkt{ X } \\
  \pi_{E} & = \bbrkt{ Y }
\end{align*}

In detail: if $\De = X_{1} : \De_{1}, \cdots , X_{n} : \De_{n}$, $E = Y_{1} : E_{1}, \cdots, Y_{m} : E_{m}$
and $\Ga = Z_{1} : \Ga_{1}, \cdots , Z_{l} : \Ga_{l}$
then the context is
\[X_{1} : \De_{1}, \cdots , X_{n} : \De_{n}, Y_{1} : E_{1}, \cdots, Y_{m} : E_{m} ,
  H_{1} : \EqCS{\Ga_{1}}{\de_{1}}{\ep_{1}}, \cdots , H_{k} : \EqCS{\Ga_{l}}{\de_{l}}{\ep_{l}}\]
and the context morphisms are those that return the relevant variables from this context.

To show the universal property, given cone $\bbrkt{Z}$
we use that judgmental equalities
$\IdObjJdg{Z}{\ze \gg \de}{\ze \gg \ep}{\Ga}$
lift
to $\ObjJdg{Z}{\SelfCS}{\EqCS{\Ga}{\ze \gg \de}{\ze \gg \ep}}$ for existence
(this fact follows from judgmental equality being a congruence);
we use \textsc{unicity} for uniqueness.

\subsection{Dependent function classes}

In the $\STLC$ case, $\la$-abstraction and function application
provide the adjunction isomorphism in the semantics

\begin{cd}
  	{\CC(\bbrkt{\Ga  \times  K_0}, \bbrkt{K_1})} && {\CC(\bbrkt{\Ga},\bbrkt{K_0 \to K_1})}
	\arrow["\lambda", shift left, from=1-1, to=1-3]
	\arrow["{\mathsf{ap}}", shift left, from=1-3, to=1-1]
\end{cd}

Let us try to consider what happens when function classes
are replaced with dependent function classes.
Instead of morphisms in the ambient category,
we need morphisms in slices,
since we are concerned with objects $\ObjJdg{\Ga}{O}{\PiCS{X}{S}{K}}$.

\begin{cd}
  	{\CC/? \, (\, {?}\, , \bbrkt{K})} && {\CC/\bbrkt{\Ga}(\id{\bbrkt{\Ga}},\bbrkt{{\PiCS{X}{S}{K}}})}
	\arrow["\textsc{pi-lam-obj}", shift left, from=1-1, to=1-3]
	\arrow["\textsc{pi-app-obj}", shift left, from=1-3, to=1-1]
\end{cd}

The introduction rule \textsc{pi-lam-obj} fill in the missing details.
We only use a special case of \textsc{pi-app-obj} here.

  {\color{syntax}
    \begin{mathpar}
      \infer[\textsc{pi-lam-obj}]{\ObjJdg{\Ga}{\LamCS{X}{S}{O}}{\PiCS{X}{S}{K}}}
        {{\color{lightgrey}\ObjJdg{\Ga}{S}{\SortCS}} & \ObjJdg{\CtxExt{\Ga}{X}{S}}{O}{K}}

      \infer[\textsc{pi-app-obj}]{\ObjJdg{\CtxExt{\Ga}{X}{S}}{\ApCS{O}{X}}{K}}
        {\ObjJdg{\CtxExt{\Ga}{X}{S}}{O}{\PiCS{X}{S}{K}} & {\color{lightgrey}\ObjJdg{\CtxExt{\Ga}{X}{S}}{X}{S}}}
    \end{mathpar}}


{\color{semantics}
\begin{cd}
  {\CC/\bbrkt{\CtxExt{\Ga}{X}{S}} \, (\id{\CtxExt{\Ga}{X}{S}} , \bbrkt{K})} && {\CC/\bbrkt{\Ga}(\id{\bbrkt{\Ga}},\bbrkt{{\PiCS{X}{S}{K}}})}
  \arrow["\textsc{pi-lam-obj}", shift left, from=1-1, to=1-3]
  \arrow["\textsc{pi-app-obj}", shift left, from=1-3, to=1-1]
\end{cd}
}

The $\be$ and $\eta$ rules for \textsc{pi} state that the maps back and forth form a bijection.
It is not yet clear what adjunction the above is a special case of.
Following the moves from before and our categorical intuition,
we first replace $\id{\bbrkt{\Ga}}$ with a more general $\CC/\bbrkt{\Ga}$-object,
a context morphism $\bbrkt{\ObjJdg{E}{\ep}{\Ga}}$.
The object is then replaced with a morphism in the slice
$\bbrkt{\ep :: f} \in \CC/\bbrkt{\Ga}(\bbrkt{\ep},\bbrkt{{\PiCS{X}{S}{K}}})$.
\begin{cd}
  	{\bbrkt{E}} && {\bbrkt{\Ga,\PiCS{X}{S}{K}}} \\
	& {\bbrkt{\Ga}}
	\arrow["{\bbrkt{\PiCS{X}{S}{K}}}", from=1-3, to=2-2]
	\arrow["{\bbrkt{\ep}}"', from=1-1, to=2-2]
	\arrow["{\bbrkt{\ep::f}}", from=1-1, to=1-3]
\end{cd}
This in particular gives some object
\[ \ObjJdg{E}{f}{\PiCS{X}{\ep S}{\brkt{(\ep :: X )K}}} \]
Weakening and applying \textsc{pi-app-obj} we obtain
\[ \ObjJdg{\CtxExt{E}{X}{\ep S}}{f X}{{(\ep ::X)K}} \]

\begin{cd}
  {\bbrkt{E,X: \ep S,Y:( \ep :: X )K}} && {\bbrkt{E,X: \ep S}} && {\bbrkt{E}} \\
  {\bbrkt{\Ga,X:S,Y:K}} && {\bbrkt{\Ga,X:S}} && {\bbrkt{\Ga}}
  \arrow["{\bbrkt{ \ep :: X ::Y}}"', from=1-1, to=2-1]
  \arrow["{\bbrkt{( \ep :: X )K}}", from=1-1, to=1-3]
  \arrow["{\bbrkt{K}}"', from=2-1, to=2-3]
  \arrow["{\bbrkt{ \ep S}}", from=1-3, to=1-5]
  \arrow["{\bbrkt{S}}"', from=2-3, to=2-5]
  \arrow["{\bbrkt{ \ep :: X }}"{description}, from=1-3, to=2-3]
  \arrow["{\bbrkt{\ep}}", from=1-5, to=2-5]
  \arrow["\lrcorner"{anchor=center, pos=0.125, rotate=0}, draw=none, from=1-3, to=2-5]
  \arrow["\lrcorner"{anchor=center, pos=0.125, rotate=0}, draw=none, from=1-1, to=2-3]
  \arrow["{\bbrkt{f X}}"', bend right, from=1-3, to=1-1]
  \arrow[dashed, from=1-3, to=2-1]
\end{cd}

The dashed arrow is then a morphism in the slice $\CC / \bbrkt{\Ga,X:S}$.
So we have (making the same argument the other way round and using $\be$ and $\eta$)
\begin{cd}
  {\CC/\bbrkt{\CtxExt{\Ga}{X}{S}} \, (\bbrkt{ \ep :: X } , \bbrkt{K})} && {\CC/\bbrkt{\Ga}(\bbrkt{\ep},\bbrkt{{\PiCS{X}{S}{K}}})}
  \arrow["\textsc{pi-lam-obj}", shift left, from=1-1, to=1-3]
  \arrow["\textsc{pi-app-obj}", shift left, from=1-3, to=1-1]
\end{cd}

This is closer to an adjunction: we can see that the left adjoint should be pullback along
$\bbrkt{S} : \bbrkt{\CtxExt{\Ga}{X}{S}} \to \bbrkt{\Ga}$.
Could ask for this bijection to work for arbitrary pullback?
The answer is \textit{not quite}.
We can replace the context extension $\bbrkt{\CtxExt{\Ga}{X}{S} \to \bbrkt{\Ga}}$ with a context morphism
$\bbrkt{\ObjJdg{\De}{\de}{\Ga}}$,
but we will only be able to find a right adjoint for pullback along $\bbrkt{\de}$ when
$\De$ \textit{only contains sorts}.
This is because \textsc{pi-cls} restricts formation of function classes to classes dependent over sorts.

\begin{mathpar}
  \infer[\textsc{ctx-sort-emp}]{\CtxSortJdg{\bullet}}{}

  \infer[\textsc{ctx-sort-ext}]{\CtxSortJdg{\CtxExt{\Ga}{X}{S}}}{\CtxSortJdg{\Ga} & \ObjJdg{\Ga}{S}{\SortCS}}
\end{mathpar}

We can also replace $\bbrkt{K}$ with a slice over $\bbrkt{\De}$ so that we have an adjunction
\begin{cd}
  {\CC/\bbrkt{\De} \, (\bbrkt{ \de }^* \bbrkt{\ep} , \bbrkt{\ze})} && {\CC/\bbrkt{\Ga}(\bbrkt{\ep},{\Pi_{\bbrkt{\de}}\bbrkt{\ze}})}
  \arrow["\textsc{pi-lam-obj}", shift left, from=1-1, to=1-3]
  \arrow["\textsc{pi-app-obj}", shift left, from=1-3, to=1-1]
\end{cd}
Briefly and roughly: Write context $Z = V_{1} : Z_{1}, \cdots, V_{n} : Z_{n}$.
A slice morphism on the left consists of objects of class $Z_{i}$ in context $E,\De,\EqCS{\Ga}{\ep}{\de}$.
Repeated $\la$-abstraction gives objects of class $\PiCS{X}{\De}{\PiCS{Y}{\EqCS{\Ga}{\ep}{\de}}{Z_{i}}}$
in context $E$, on the right hand side.
So we should define ${\Pi_{\bbrkt{\de}}\bbrkt{\ze}}$ such that its object part is context
\[
  \bbrkt{V_{1} : \PiCS{X}{\De}{\PiCS{Y}{\EqCS{\Ga}{\ep}{\de}}{Z_{1}},\cdots, V_{n} : \PiCS{X}{\De}{\PiCS{Y}{\EqCS{\Ga}{\ep}{\de}}{Z_{n}}}}}
\]
and its morphism part comes from each $Z_{i}$ being a class in context $\De$.
The proof that it is an adjunction is in the same vein as before.

\subsection{Universes}
It is not the case that the category of contexts is a locally cartesian closed category.
In particular, we have pullbacks but not $\Pi$s for arbitrary morphisms in the category.
However, the full subcategory of sort-only contexts \textit{is} locally cartesian closed.
Note that although this subcategory of the syntactic category is an LCCC,
it is not the case that locally cartesian closed categories are
fit for semantic purposes due to the substitution/pullback issue addressed later.



% \begin{align*}
%   & \ObjJdg{\Ga}{S}{\SortCS}          & \rightsquigarrow & \ObjJdg{\Ga}{\De}{\SortCS} \\
%   & \CtxExt{\Ga}{X}{S}                & \rightsquigarrow & \De \\
%   & \ClsJdg{\CtxExt{\Ga}{X}{S}}{K}    & \rightsquigarrow & \ObjJdg{E}{\}{\De} \\
%   & \ObjJdg{\CtxExt{\Ga}{X}{S}}{O}{K} & \rightsquigarrow & \ObjJdg{\De}{\si}{E} \\
%   & \ObjJdg{\CtxExt{\Ga}{X}{S}}{O}{K}
% \end{align*}

% In this category the objects are contexts.
% In the case of $\STLC$ we could equivalently see contexts as closed classes
% $\ClsJdg{\bullet}{K}$ by taking the product of all the types appearing in the context.




\bibliography{refs}{}
\bibliographystyle{alpha}
\end{document}
