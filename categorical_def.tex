\section{Categorical definitions}

\begin{dfn}[Cartesian closed category (CCC)]
  \link{dfn_CCC}
  We say a category $\CC$ is cartesian closed when
  \begin{itemize}
    \item $\CC$ has finite products
    \item For each $\CC$-object $A$, the product with $A$ functor $\times A : \CC \to \CC$ has a right adjoint.
  \end{itemize}
  We denote the right adjoint by $[A,-]$ and call it internal hom.
\end{dfn}

\begin{dfn}[Locally cartesian closed category (LCCC)]
  We say a category $\CC$ is locally cartesian closed when either of the following equivalent definitions hold
  \begin{enumerate}
    \item Every slice category $\CC / A$ over a $\CC$-object $A$ is cartesian closed.
    \item $\CC$ has pullbacks, and for each morphism $f : B \to A$ in $\CC$,
          the base change (pullback along $f$) $f^{*} : \CC / A \to \CC / B$ has a right adjoint.
  \end{enumerate}
  We denote the right adjoint to base change by $\Pi_{f}$.
\end{dfn}

We do not show that these two definitions are equivalent.
The idea is that existence of pullbacks correspond to existence of products in slices
and $\Pi$s correspond to internal homs in slices.

% https://q.uiver.app/?q=WzAsNSxbMywwLCJcXENDL0EiXSxbMywxLCJcXENDL0IiXSxbMywyLCJcXENDL0EiXSxbMCwxLCJCIl0sWzEsMSwiQSJdLFswLDEsImZeKiIsMix7Im9mZnNldCI6Mn1dLFsxLDAsIlxcUGlfZiIsMix7Im9mZnNldCI6Mn1dLFsxLDIsIlxcU2lfZiIsMix7Im9mZnNldCI6Mn1dLFsyLDEsImZeKiIsMix7Im9mZnNldCI6Mn1dLFswLDIsIlxcdGltZXMgZiIsMSx7ImN1cnZlIjo0fV0sWzIsMCwiLV5mIiwxLHsiY3VydmUiOjV9XSxbMyw0LCJmIl0sWzUsNiwiIiwyLHsibGV2ZWwiOjEsInN0eWxlIjp7Im5hbWUiOiJhZGp1bmN0aW9uIn19XV0=
\[\begin{tikzcd}
	&&& {\CC/A} \\
	B & A && {\CC/B} \\
	&&& {\CC/A}
	\arrow[""{name=0, anchor=center, inner sep=0}, "{f^*}"', shift right=2, from=1-4, to=2-4]
	\arrow[""{name=1, anchor=center, inner sep=0}, "{\Pi_f}"', shift right=2, from=2-4, to=1-4]
	\arrow[""{name=2, anchor=center, inner sep=0}, "{\Si_f}"', shift right=2, from=2-4, to=3-4]
	\arrow[""{name=3, anchor=center, inner sep=0}, "{f^*}"', shift right=2, from=3-4, to=2-4]
	\arrow["{\times f}"{description}, bend left = {- 60}, from=1-4, to=3-4]
	\arrow["{-^f}"{description}, bend left = {- 60}, from=3-4, to=1-4]
	\arrow["f", from=2-1, to=2-2]
	\arrow["\dashv"{anchor=center}, draw=none, from=0, to=1]
	\arrow["\dashv"{anchor=center}, draw=none, from=2, to=3]
\end{tikzcd}\]

In the above $\Si_{f}$ is the functor that composes the map from the slice with $f$.

\begin{dfn}[Display]
  \link{dfn_display_map}
  Let $\SS$ be a set of morphisms in category $\CC$.
  We say $\SS$ is a \textit{set of display maps} when for any compatible
  pair of morphisms $f : A \to B$ in $\CC$ and $g : C \to B$ in $\SS$
  the pullback exists and $f^{*}g \in \SS$.

  We say $\SS$ is a \textit{display structure} when we have an operation that
  takes a pair of morphisms $f : A \to B$ in $\CC$ and $g : C \to B$ in $\SS$
  and returns a pullback such that $f^{*}g \in \SS$.
\end{dfn}

% \begin{dfn}
%   Let $\SS$ be a set of morphisms in a category $\CC$.
%   We say the pair $\SS$ is kind-of-representable (KoR) when
%   % Uemura \cite{uemura2019general} defines the pair $(\CC,\SS)$ to form a
%   % representable map category when $\CC$ is finitely complete and
%   \begin{itemize}
%     \item $\SS$ forms a subcategory of $\CC$
%     \item $\SS$ is a display structure
%     \item
%   \end{itemize}
% \end{dfn}

The following definition is similar in spirit to both the definition of
\textit{universes in a topos} from \cite{streicher} and also
\textit{representable map categories} from Uemura \cite{uemura2019general}.

\begin{dfn}[Universe of representable maps]
  \link{dfn_urm}
  Let $\CC$ be a category, and $\SS$ a set of morphisms in $\CC$.
  We may denote morphisms in $\SS$ by $S : \Ga,S \to \Ga$ or $\si : \De \to \Ga$,
  though this is just notation.
  We say $\SS$ is an universe of representable maps (URM) when
  \begin{enumerate}
    \item $\SS$ is stable under $\CC$-pullbacks (strictly),
          i.e. if $\de : \De \to \Ga$ is in $\CC$ and $S : \Ga,S \to \Ga$ is in $\SS$
          and we have a pair of morphisms $S^{*} \de : \De,\de S \to \Ga,S$ in $\CC$ and
          $\de^* S : \De,\de S \to \De$ in $\SS$ such that
          \[\begin{tikzcd}
            {\De,\de^* S} & {\Ga,S} \\
            \De & \Ga
            \arrow["S", from=1-2, to=2-2]
            \arrow["{\de^* S}"', from=1-1, to=2-1]
            \arrow["\de"', from=2-1, to=2-2]
            \arrow["S^{*} \de", from=1-1, to=1-2]
            \arrow["\lrcorner"{anchor=center, pos=0.125}, draw=none, from=1-1, to=2-2]
          \end{tikzcd}\]
          Furthermore, this preserves identities and compositions from $\CC$ strictly, so that
          \[ \id{\Ga}^{*} S = S \text{ and } S^{*} \id{\Ga} = \id{\Ga,S}\]
          and
          \[ \ep^{*} (\de^{*} S) = (\de \circ \ep)^{*} S \text{ and } (S^{*}\de)^{*}\ep = S^{*}(\de \circ \ep) \]
          from which it follows that
          $S^{*} : \CC / \Ga \to \CC / \Ga,S$ is a functor.
    \item $\SS$ is closed under identities, i.e. if $\si : \De \to \Ga$ is in $\SS$ then
          so are $\id{\De}$ and $\id{\Ga}$.
    \item $\SS$ is closed under composition (aka $\Si$s), i.e. if
          \[\begin{tikzcd}
              {\Ga,S,T} & {\Ga,S} & \Ga
              \arrow["T", from=1-1, to=1-2]
              \arrow["\in \SS"', from=1-1, to=1-2]
              \arrow["S", from=1-2, to=1-3]
              \arrow["\in \SS"', from=1-2, to=1-3]
            \end{tikzcd}
          \quad \quad \text{ then } \quad \quad \Si_{S} T := S \circ T \in \SS\]
    \item $\SS$ is closed under $\Pi$s, i.e. if $S : \Ga,S \to \Ga$ is in $\SS$
          then the pullback functor $S^{*} : \CC / \Ga \to \CC / \Ga,S$ has a right
          adjoint $\Pi_{S} : \CC / \Ga,S \to \CC / \Ga$, and
            \[\begin{tikzcd}
              {\Ga,S,T} & {\Ga,S} & \Ga
              \arrow["T", from=1-1, to=1-2]
              \arrow["\in \SS"', from=1-1, to=1-2]
              \arrow["S", from=1-2, to=1-3]
              \arrow["\in \SS"', from=1-2, to=1-3]
            \end{tikzcd}
          \quad \quad \text{ then } \quad \quad \Pi_{S} T \in \SS\]
  \end{enumerate}

  Note that the first four conditions combined result in $\SS$ forming a subcategory of $\CC$ that is an LCCC.
  We can think of a URM as the smallest LCCC stable under $\CC$-pullbacks, generated by some maps.
  The main difference of this definition with that of a representable map category is
  dropping the condition that $\CC$ is finitely complete.
  We address the differences with the definition of a universe in a topos \linkto{dfn_universe}{later}.
  Note that in particular any URM is a display structure.
\end{dfn}
