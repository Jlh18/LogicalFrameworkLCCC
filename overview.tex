\section{Overview}

\subsection{Terminology}

We follow \cite{harper2021equational} and use the word ``class'' for what is usually called ``type'' and
the word ``object'' for what is usually called ``term''.
This is to distinguish between the meta-level class in the logical framework
and types internal to the type theory defined in the logical framework.
To not confuse this with the categorical notion of object,
we will write $\CC$-object for an object in a category $\CC$.

\subsection{Overview of this note}

In \linkto{sec_2}{section 2}, we present the syntax for
an \textit{equational logical framework} (LF for short).
In \linkto{sec_3}{section 3}, we set out the main categorical
definitions that are relevant to the discussion.
In \linkto{sec_4}{section 4}, we describe the category of contexts
for the LF.
The main goal is examine this category of context via definitions
in the literature.
In particular I will identify two subcategories $\SS_{\Pi}$ and $\SS_{\Pi,=}$
of the category of contexts, and identify which definitions these satisfy.

\subsection{Definitions in the literature}
There are many closely related definitions that are useful in analyzing the category of contexts.
Let us denote with $\CC$ the category of contexts for our LF,
and fix two subcategories (or collection of maps in $\CC$) $\SS_{\Pi}$ and $\SS_{\Pi,=}$ which we define later.
\begin{itemize}
  \item $\CC$ is \textit{not finitely complete}, in particular $\CC$ does not have all pullbacks.
  \item Because $\CC$ is not finitely complete, $\CC$ is
        \textit{not} a \textit{representable map category}
        as in Uemura \cite{uemura2019general}.
        However, by {\color{syntax} adding an equality class for objects in classes}
        \[\color{syntax}
            \infer[\textsc{eq-cls}]{\ClsJdg{\Ga}{\EqCS{K}{O_0}{O_1}}}
            {\ObjJdg{\Ga}{O_0}{K} & \ObjJdg{\Ga}{O_1}{K}}
        \]
        or equivalently {\color{semantics} taking the finite completion $\bar{\CC}$},
        we can show that both $(\bar{\CC},\SS_{\Pi})$ and $(\bar{\CC},\SS_{\Pi,=})$ are representable map categories.
  \item Because $\CC$ is not finitely complete, $\CC$ is not locally Cartesian closed.
        Furthermore, adding an equality class/taking the finite completion $\bar{\CC}$
        still does not give a locally Cartesian closed category due to lack of
        {\color{syntax} global dependent function classes}/{\color{semantics}$\Pi$s for all $\bar{\CC}$-morphisms}.
        Then {\color{syntax} adding a global dependent function class (and equality classes for objects in classes)}
        is equivalent to {\color{semantics} taking the local Cartesian closure as in Gratzer \cite{gratzer2021syntactic}},
        \[\color{syntax}
          \infer[\textsc{pi-cls}]{\ClsJdg{\Ga}{\PiCS{X}{K_{0}}{K_{1}}}}
            {\ClsJdg{\Ga}{K_{0}} & \ClsJdg{\CtxExt{\Ga}{X}{K_{0}}}{K_{1}}}
        \]
  \item Both $\SS_{\Pi}$ and $\SS_{\Pi,=}$ are \textit{\linkto{dfn_display_map}{display structures}}
        (hence sets of \textit{display maps}) as in Taylor \cite{Taylor1999}.
        Related to the first point about representable map categories,
        $\SS_{\Pi}$ and $\SS_{\Pi,=}$ are locally Cartesian closed.
  \item Both $\SS_{\Pi}$ is similar in spirit to a universe (originally universe in a topos) as in Streicher \cite{streicher}.
        The main differences are $1.$ that {\color{semantics} monos} are not really related to our syntax;
        they correspond to ``propositions'' in the sense of bundles with subsingleton fibers.
        And $2.$ that our syntax has no {\color{syntax} dependent pair class ($\Si$-type) or unit type},
        which correspond to {\color{semantics} composition and identity respectively} in the presence of $\ElCS$,
        a display map classifying context extension by a sort.
\end{itemize}

For most definitions there will be a strict/structure version as well as a non-strict/proposition version,
due to coherence issues for pullback.
For example, display structures in contrast to classes of displays.
To tackle such issues for general semantics,
we can use for example categories with families \cite{castellan2020categories},
categories with attributes \cite{JACOBS1993169},
or natural models \cite{awodey2017natural}.
One can verify that the category of contexts $\CC$ can be used in any of these semantics.


\subsection{Acknowledgements}
Many thanks to Bob Harper, Astra Kolomatskaia, Kian Cho, Fernando Larrain Langlois and Steve Awodey
for their insights.
