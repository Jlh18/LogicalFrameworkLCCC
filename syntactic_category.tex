\section{The category of contexts}

I order to find categorical semantics for a type theory
it is useful to study the category of contexts.
In this section we construct the category of contexts.
We will call this category $\CC$ for the rest of this section.

We begin by recalling $\STLC$,
simply typed $\la$-calculus with only unit, product and function classes.
And use intuition of its categorical semantics to motivate those for logical frameworks.
Recall that $\STLC$ can be interpreted in a \linkto{dfn_CCC}{CCC} $\DD$ by
taking classes $K$ as $\DD$-objects $\bbrkt{K}$,
contexts $\CtxExt{X_{1}:K_{1},\dots}{X_{n}}{K_{n}}$
as products $\bbrkt{K_{1}} \tdt \bbrkt{K_{n}}$,
and objects $\ObjJdg{\Ga}{K}{O}$ as morphisms $\bbrkt{O} : \bbrkt{\Ga} \to \bbrkt{K}$.
This seems quite natural, since product classes become products,
and function classes become internal homs in the semantics.
If we are to immitate this for $\LF{\Si}$,
we would only be able to talk about \textit{closed classes} $\ClsJdg{\bullet}{K}$,
since in $\STLC$ we don't have dependent classes $\ClsJdg{\Ga}{K}$.
However, in the presence of dependent pairs (aka $\Si$-types) we
could view a context $\Ga = \CtxExt{X_{1}:K_{1},\dots}{X_{n}}{K_{n}}$
as a closed class $\Si_{\Ga} := \Si_{X_{1}:K_{1}} \cdots \Si_{X_{n-1}:K_{n-1}} K_{n}$.
Then any dependent class can be viewed as a context, which can be viewed as a closed class.

\subsection{Contexts and dependent classes}

Instead of including dependent pairs,
let us view dependent classes $\ClsJdg{\Ga}{K}$ as extended contexts
$\CtxExt{\Ga}{X}{K}$, and take \textit{contexts}
(\linkto{up_to_jdg_eq}{up to judgmental equality}) as $\CC$-objects,
and \textit{as a consequence} dependent classes as $\CC$-objects.

\begin{multicols}{2}
  \color{syntax}{
  \[\CtxJdg{\Ga}\]}

  \color{semantics}{
  \[\bbrkt{\Ga} \in \CC\]
  }
\end{multicols}

It is then natural to define morphisms of contexts (aka a substitution)
(\linkto{up_to_jdg_eq}{up to judgmental equality}),
denoted $\ObjJdg{\Ga}{\si}{\De}$,
consisting of the data of objects

\begin{align*}
  &\ObjJdg{\Ga}{\si_{1}}{K_{1}}\\
  &\ObjJdg{\Ga}{\si_{2}}{\Subst{\si_{1}}{X_{1}}{K_{2}}}\\
  &\vdots & \text{ when } \De = \CtxExt{X_{1}:K_{1}}{\cdots, X_{n}}{K_{n}}\\
  &\ObjJdg{\Ga}{\si_{n}}{\Subst{\si_{n-1}}{X_{n-1}}{\cdots \Subst{\si_{1}}{X_{1}}{K_{n}}}}
\end{align*}

Such a substitution corresponds to a closed object of the closed class
$\Si_{\Ga} \to \Si_{\De}$ in the presence of dependent pairs.

\begin{multicols}{2}
  \color{syntax}{
  \[\ObjJdg{\Ga}{\si}{\De}\]}

  \color{semantics}{
  \[\bbrkt{\si} : \bbrkt{\Ga} \to \bbrkt{\De}\]
  }
\end{multicols}

The empty context gives an object $\bbrkt{\bullet}$,
and for any context $\Ga$ there is exactly one substitution $\ObjJdg{\Ga}{\si}{\bullet}$,
given by the data of no objects.
This makes $\bbrkt{\bullet}$ the terminal object.

\begin{multicols}{2}
  \color{syntax}{
  \[\infer[\textsc{ctx-emp}]{\CtxJdg{\bullet}}{}\]}

  \color{semantics}{
  \[\TERM_{\CC} \in \CC\]
  }
\end{multicols}

For context extension, given $\ClsJdg{\Ga}{K}$,
we obtain a substitution $\ObjJdg{\CtxExt{\Ga}{X}{K}}{\si}{\Ga}$
by $\ObjJdg{\CtxExt{\Ga}{X}{K}}{X_{i}}{K_{i}}$ for each $X_{i} : K_{i} \in \Ga$.
Then let's take the interpretation of $K$ to be a $\CC/\bbrkt{\Ga}$-object,
where $\CC/\bbrkt{\Ga}$ is the slice category.

\[ \bbrkt{\ClsJdg{\Ga}{K}} := \bbrkt{\CtxExt{\Ga}{X}{K}} \text{ and } \bbrkt{\ObjJdg{\CtxExt{\Ga}{X}{K}}{\si}{\Ga}}\]

We can informally identify use the slice-object $\bbrkt{\ClsJdg{\Ga}{K}}$ to denote the underlying morphism.

\begin{multicols}{2}
  \color{syntax}{
  \[\infer[\textsc{ctx-ext}]{\CtxJdg{\CtxExt{\Ga}{X}{K}}}{\ClsJdg{\Ga}{K}}\]}


  \color{semantics}{
    \begin{cd}
     & \\
  	{\bbrkt{\CtxExt{\Ga}{X}{K}}} && {\bbrkt{\Ga}}
	\arrow["{\bbrkt{\ClsJdg{\Ga}{K}}}", from=2-1, to=2-3]
  \end{cd}
  }
\end{multicols}

\subsection{Judgmental Equality}
\link{up_to_jdg_eq}
Technically we should we taking $\CC$-objects to be the
set of contexts quotiented by the equivalence relation induced by judgmental equality on classes,
i.e. $\CtxExt{{X_{1} : K_{1}},\cdots}{X_{n}}{K_{n}} \sim \CtxExt{{Y_{1} : K'_{1}},\cdots}{Y_{n}}{K'_{n}}$
if and only if
\begin{align*}
  \IdClsJdg{\bullet}{K_{1}}{K'_{1}}\\
  \IdClsJdg{X_{1} : K_{1}}{K_{2}}{K'_{2}}\\
  \vdots \quad \quad \quad \quad \quad \quad  \\
  \IdClsJdg{\CtxExt{{X_{1} : K_{1}},\cdots}{X_{n-1}}{K_{n-1}}}{K_{n}}{K'_{n}}
\end{align*}
We always denote an equivalence class using a representative $\bbrkt{\Ga}$,
and omit the quotient-related proofs.

We should also be taking $\CC$-morphisms from $\bbrkt{\Ga}$ to $\bbrkt{\De}$
to be substitutions quotiented by the equivalence relation induced by judgmental equality on terms,
i.e. $\ObjJdg{\Ga}{\si}{\De} \sim \ObjJdg{E}{\rho}{Z}$ if and only if $\bbrkt{\Ga} = \bbrkt{E}$ and
$\bbrkt{\De} = \bbrkt{Z}$ and
\begin{align*}
  &\IdObjJdg{\Ga}{\si_{1}}{\rho_{1}}{K_{1}}\\
  &\IdObjJdg{\Ga}{\si_{2}}{\rho_{2}}{\Subst{\si_{1}}{X_{1}}{K_{2}}}\\
  &\vdots & \text{ when } \De = \CtxExt{X_{1}:K_{1}}{\cdots, X_{n}}{K_{n}}\\
  &\IdObjJdg{\Ga}{\si_{n}}{\rho_{n}}{\Subst{\si_{n-1}}{X_{n-1}}{\cdots \Subst{\si_{1}}{X_{1}}{K_{n}}}}
\end{align*}
Likewise, we always denote an equivalence class using a representative $\bbrkt{\si}$,
and omit the quotient-related proofs.

\subsection{Objects}

In the $\STLC$ case we interpreted objects
$\ObjJdg{\Ga}{O}{K}$ as morphisms $\bbrkt{O} : \bbrkt{\Ga} \to \bbrkt{K}$.
Since the classes are now dependent,
and the object is situated in some context,
we should upgrade this to a morphism in the slice $\CC / \bbrkt{\Ga}$.

\begin{multicols}{2}

    \[\color{syntax}\ObjJdg{\Ga}{O}{K}\]


  \color{semantics}{
  \begin{cd}
     {\bbrkt{\Ga}} && {\bbrkt{\Ga,X:K}} \\
    & {\bbrkt{\Ga}}
    \arrow["{\bbrkt{\ObjJdg{\Ga}{O}{K}}}", from=1-1, to=1-3]
    \arrow["{\id{\bbrkt{\Ga}}}"', from=1-1, to=2-2]
    \arrow["{\bbrkt{\ClsJdg{\Ga}{K}}}", from=1-3, to=2-2]
  \end{cd}
  }
\end{multicols}

We can call $\bbrkt{\ObjJdg{\Ga}{O}{K}}$
a \textit{section} of the \textit{bundle} $\bbrkt{\ClsJdg{\Ga}{K}}$.
This is a bundle in the sense that for any collection of objects $O_{1},\cdots,O_{n}$ in the classes from $\Ga$,
there is a closed class \[\ClsJdg{\bullet}{[O_{n}/X_{n}] \cdots \Subst{O_{1}}{X_{1}}{K}}\]
which is the \textit{fiber} over that collection of objects.
A section of the bundle then picks out for any such collection of objects,
an object in the \textit{fiber}
\[\ObjJdg{\bullet}{[O_{n}/X_{n}] \cdots \Subst{O_{1}}{X_{1}}{O}}{[O_{n}/X_{n}] \cdots \Subst{O_{1}}{X_{1}}{K}}\]
\begin{multicols}{2}

    \[\color{syntax}  \infer[\textsc{var-obj}]{\ObjJdg{\Ga}{X}{K}}
    {\CtxJdg{\Ga} & X:K \in \Ga}\]


  \color{semantics}{
  \begin{cd}
     {\bbrkt{\Ga}} && {\bbrkt{\Ga,Y:K}} \\
    & {\bbrkt{\Ga}}
    \arrow["{\bbrkt{\ObjJdg{\Ga}{X}{K}}}", from=1-1, to=1-3]
    \arrow["{\id{\bbrkt{\Ga}}}"', from=1-1, to=2-2]
    \arrow["{\bbrkt{\ClsJdg{\Ga}{K}}}", from=1-3, to=2-2]
  \end{cd}
  }
\end{multicols}

We may also write $\bbrkt{K}$ to mean $\bbrkt{\ClsJdg{\Ga}{K}}$ for brevity.

\subsection{Substitution action on classes and objects}

Context morphisms (aka substitutions) are dependent tuples of objects,
just as contexts are dependent tuples of classes.
They can be {\color{semantics} composed}, which we can view as {\color{syntax} sequential substitution}.
They have an {\color{syntax} action on dependent classes},
which we will show corresponds to {\color{semantics} pullback of bundles}.
They have an {\color{syntax} action on objects} (more generally {\color{syntax} action on other substitutions}),
which we will show corresponds to {\color{semantics} pullback of sections}
(more generally {\color{semantics} existence of all pullbacks}).

First let us suppose $\ObjJdg{\Ga}{\si}{\De}$ is a substitution and $\ClsJdg{\De}{K}$.
Then we can form $\ClsJdg{\Ga}{\si K}$,
where
\[ \si K := \Subst{\si_{n}}{X_{n}}{ \cdots \Subst{\si_{1}}{X_{1}}{K}} \quad \text{ and }
  \quad \De = \CtxExt{X_{1}:K_{1}}{\cdots, X_{n}}{K_{n}}\]

This automatically gives us a substitution $\ObjJdg{\CtxExt{\Ga}{X}{\si K}}{\si :: X}{{\CtxExt{\De}{X}{K}}}$.
This says we have a pullback

\begin{multicols}{2}

\[\color{syntax} \infer{\ClsJdg{\Ga}{\si K}}{\ObjJdg{\Ga}{\si}{\De} & & \ClsJdg{\De}{K} } \]

%% this admissible judgment needs a name

{\color{semantics}
\begin{cd}
  {\bbrkt{\CtxExt{\Ga}{X}{\si K}}} && {\bbrkt{\CtxExt{\De}{X}{K}}} \\
  \\
  {\bbrkt{\Ga}} && {\bbrkt{\De}}
  \arrow["{\bbrkt{\si}}"', from=3-1, to=3-3]
  \arrow["{\bbrkt{K}}", from=1-3, to=3-3]
  \arrow["{\bbrkt{\si :: X}}", dashed, from=1-1, to=1-3]
  \arrow["{\bbrkt{\si K}}"', dashed, from=1-1, to=3-1]
  \arrow["\lrcorner"{anchor=center, pos=0.125}, draw=none, from=1-1, to=3-3]
\end{cd}}

\end{multicols}

To check the diagram commutes, we need to check judgmental equality component-wise
(opening up the equivalence class definition).
For the universal property we need to apply (syntactic) substitution on objects.

Now suppose $\ObjJdg{\De}{O}{K}$.
Then we can form $\ObjJdg{\Ga}{\si O}{\si K}$,
where
\[ \si O := \Subst{\si_{n}}{X_{n}}{ \cdots \Subst{\si_{1}}{X_{1}}{O}}\]

This says our pullback diagram extends

\newpage

\begin{multicols}{2}

\[\color{syntax} \infer{\ObjJdg{\Ga}{\si O}{\si K}}{\ObjJdg{\Ga}{\si}{\De} & & \ObjJdg{\De}{O}{K} } \]

%% this admissible judgment needs a name

{\color{semantics}
\begin{cd}
  	{\bbrkt{\Ga}} && {\bbrkt{\De}} \\
	\\
	{\bbrkt{\CtxExt{\Ga}{X}{\si K}}} && {\bbrkt{\CtxExt{\De}{X}{K}}} \\
	\\
	{\bbrkt{\Ga}} && {\bbrkt{\De}}
	\arrow["{\bbrkt{\si}}"', from=5-1, to=5-3]
	\arrow["{\bbrkt{K}}", from=3-3, to=5-3]
	\arrow["{\bbrkt{\si :: X}}", from=3-1, to=3-3]
	\arrow["{\bbrkt{\si K}}"', from=3-1, to=5-1]
	\arrow["\lrcorner"{anchor=center, pos=0.125}, draw=none, from=3-1, to=5-3]
	\arrow["{\bbrkt{\si O}}"', dashed, from=1-1, to=3-1]
	\arrow["{\bbrkt{O}}", from=1-3, to=3-3]
	\arrow["{\bbrkt{\si}}", from=1-1, to=1-3]
	\arrow["\lrcorner"{anchor=center, pos=0.125}, draw=none, from=1-1, to=3-3]
\end{cd}}
\end{multicols}

Now we can use substitution on classes and objects to define composition of context morphisms.
Given two substitutions $\ObjJdg{\Ga_{0}}{\si}{\Ga_{1}}$ and $\ObjJdg{\Ga_{1}}{\rho}{\Ga_{2}}$
we define the composition $\ObjJdg{\Ga_{0}}{\si \gg \rho}{\Ga_{2}}$
by giving

\begin{align*}
  &\ObjJdg{\Ga_{0}}{\si{\rho_{1}}}{\si K_{1}}\\
  &\ObjJdg{\Ga_{0}}{\si\rho_{2}}{\si \Subst{\rho_{1}}{X_{1}}{K_{2}}} \\
  &\vdots & \text{ where } & \Ga_{2} = \CtxExt{X_{1}:K_{1}}{\cdots, X_{n}}{K_{n}}\\
  &\ObjJdg{\Ga_{0}}{\si \rho_{n}}{\si \Subst{\rho_{n-1}}{X_{n-1}}{\cdots \Subst{\rho_{1}}{X_{1}}{K_{n}}}}
\end{align*}

So we have

\begin{multicols}{2}
  \[\color{syntax} \ObjJdg{\Ga_{0}}{\si \gg \rho}{\Ga_{2}}\]

   \[ \color{semantics}
    \bbrkt{\rho} \circ \bbrkt{\si} : \bbrkt{\Ga_0} \to \bbrkt{\Ga_2} \]
\end{multicols}

There is a techincal issue here about substitution not working out in a general LCCC,
because of pullback only being unique up to isomorphism.
However, in the syntax category we have ``strict pullbacks''.


\subsection{Strict pullbacks}
\link{strict_pullbacks}

The previous construction of the pullback $\bbrkt{K} \mapsto \bbrkt{\si} \mapsto (\bbrkt{\si K},\bbrkt{\si :: X})$
defines a strictly equal diagram.
\[ \bbrkt{\si}^{*} \bbrkt{\rho}^{*} \bbrkt{K} = \bbrkt{\si (\rho K)} =
   \bbrkt{(\si \gg \rho) K} = \brkt{\bbrkt{\rho} \circ \bbrkt{\si}}^{*} \bbrkt{K}\]

% https://q.uiver.app/?q=WzAsOSxbMiwxLCJcXEdhXzEsXFxyaG8gSyJdLFsyLDIsIlxcR2FfMSJdLFszLDEsIlxcR2FfMixLIl0sWzMsMiwiXFxHYV8yIl0sWzEsMSwiXFxHYV8wLFxcc2kgXFxyaG8gSyJdLFsxLDIsIlxcR2FfMCJdLFswLDBdLFswLDIsIlxcR2FfMCJdLFswLDEsIlxcR2FfMCwoXFxzaSBcXGdnIFxccmhvKSBLIl0sWzAsMSwiXFxyaG8gSyJdLFswLDJdLFsyLDMsIksiXSxbMSwzLCJcXHJobyIsMl0sWzAsMywiIiwxLHsic3R5bGUiOnsibmFtZSI6ImNvcm5lciJ9fV0sWzQsMF0sWzQsNSwiXFxzaSBcXHJobyBLIiwyXSxbNCwxLCIiLDEseyJzdHlsZSI6eyJuYW1lIjoiY29ybmVyIn19XSxbNSwxLCJcXHNpIiwyXSxbNyw1LCI9IiwxXSxbOCw3LCIoXFxzaSBcXGdnIFxccmhvKSBLID0iXSxbOCw0LCI9IiwxXSxbOCw1LCIiLDEseyJzdHlsZSI6eyJuYW1lIjoiY29ybmVyIn19XV0=
\[\begin{tikzcd}
	{} \\
	{\Ga_0,(\si \gg \rho) K} & {\Ga_0,\si \rho K} & {\Ga_1,\rho K} & {\Ga_2,K} \\
	{\Ga_0} & {\Ga_0} & {\Ga_1} & {\Ga_2}
	\arrow["{\rho K}", from=2-3, to=3-3]
	\arrow[from=2-3, to=2-4]
	\arrow["K", from=2-4, to=3-4]
	\arrow["\rho"', from=3-3, to=3-4]
	\arrow["\lrcorner"{anchor=center, pos=0.125}, draw=none, from=2-3, to=3-4]
	\arrow[from=2-2, to=2-3]
	\arrow["{\si \rho K}" {name=0}, from=2-2, to=3-2]
	\arrow["\lrcorner"{anchor=center, pos=0.125}, draw=none, from=2-2, to=3-3]
	\arrow["\si"', from=3-2, to=3-3]
	\arrow["{=}"{description}, from=3-1, to=3-2]
	\arrow["{(\si \gg \rho) K}"' {name=1}, from=2-1, to=3-1]
	\arrow["{=}"{description}, from=2-1, to=2-2]
  \arrow["{=}"{description}, draw=none, from=1, to=0]
	\arrow["\lrcorner"{anchor=center, pos=0.125}, draw=none, from=2-1, to=3-2]
\end{tikzcd}\]


Note that in general categories (including LCCCs) the pullback along a composition is only
necessarily isomorphic to double pullback.

\subsection{Dependent function classes}

In the $\STLC$ case, $\la$-abstraction and function application
provide the adjunction isomorphism in the semantics

\begin{cd}
  	{\CC(\bbrkt{\Ga  \times  K_0}, \bbrkt{K_1})} && {\CC(\bbrkt{\Ga},\bbrkt{K_0 \to K_1})}
	\arrow["\lambda", shift left, from=1-1, to=1-3]
	\arrow["{\mathsf{ap}}", shift left, from=1-3, to=1-1]
\end{cd}

Let us try to consider what happens when function classes
are replaced with dependent function classes.
Instead of morphisms in the ambient category,
we need morphisms in slices,
since we are concerned with objects $\ObjJdg{\Ga}{O}{\PiCS{X}{S}{K}}$.

\begin{cd}
  	{\CC/? \, (\, {?}\, , \bbrkt{K})} && {\CC/\bbrkt{\Ga}(\id{\bbrkt{\Ga}},\bbrkt{{\PiCS{X}{S}{K}}})}
	\arrow["\textsc{pi-lam-obj}", shift left, from=1-1, to=1-3]
	\arrow["\textsc{pi-app-obj}", shift left, from=1-3, to=1-1]
\end{cd}

The introduction rule \textsc{pi-lam-obj} fill in the missing details.
We only use a special case of \textsc{pi-app-obj} here.

  {\color{syntax}
    \begin{mathpar}
      \infer[\textsc{pi-lam-obj}]{\ObjJdg{\Ga}{\LamCS{X}{S}{O}}{\PiCS{X}{S}{K}}}
        {{\color{lightgrey}\ObjJdg{\Ga}{S}{\SortCS}} & \ObjJdg{\CtxExt{\Ga}{X}{S}}{O}{K}}

      \infer[\textsc{pi-app-obj}]{\ObjJdg{\CtxExt{\Ga}{X}{S}}{\ApCS{O}{X}}{K}}
        {\ObjJdg{\CtxExt{\Ga}{X}{S}}{O}{\PiCS{X}{S}{K}} & {\color{lightgrey}\ObjJdg{\CtxExt{\Ga}{X}{S}}{X}{S}}}
    \end{mathpar}}


{\color{semantics}
\begin{cd}
  {\CC/\bbrkt{\CtxExt{\Ga}{X}{S}} \, (\id{\CtxExt{\Ga}{X}{S}} , \bbrkt{K})} && {\CC/\bbrkt{\Ga}(\id{\bbrkt{\Ga}},\bbrkt{{\PiCS{X}{S}{K}}})}
  \arrow["\textsc{pi-lam-obj}", shift left, from=1-1, to=1-3]
  \arrow["\textsc{pi-app-obj}", shift left, from=1-3, to=1-1]
\end{cd}
}

The $\be$ and $\eta$ rules for \textsc{pi} state that the maps back and forth form a bijection.
It is not yet clear what adjunction the above is a special case of.

Following the moves from before and our categorical intuition,
we first replace $\id{\bbrkt{\Ga}}$ with a more general $\CC/\bbrkt{\Ga}$-object,
a context morphism $\bbrkt{\ObjJdg{E}{\ep}{\Ga}}$.
The object is then replaced with a morphism in the slice
$\bbrkt{\ep :: f} \in \CC/\bbrkt{\Ga}(\bbrkt{\ep},\bbrkt{{\PiCS{X}{S}{K}}})$.
\begin{cd}
  	{\bbrkt{E}} && {\bbrkt{\Ga,\PiCS{X}{S}{K}}} \\
	& {\bbrkt{\Ga}}
	\arrow["{\bbrkt{\PiCS{X}{S}{K}}}", from=1-3, to=2-2]
	\arrow["{\bbrkt{\ep}}"', from=1-1, to=2-2]
	\arrow["{\bbrkt{\ep::f}}", from=1-1, to=1-3]
\end{cd}
This in particular gives some object
\[ \ObjJdg{E}{f}{\PiCS{X}{\ep S}{\brkt{(\ep :: X )K}}} \]
Weakening and applying \textsc{pi-app-obj} we obtain
\[ \ObjJdg{\CtxExt{E}{X}{\ep S}}{f X}{{(\ep ::X)K}} \]

\begin{cd}
  {\bbrkt{E,X: \ep S,Y:( \ep :: X )K}} && {\bbrkt{E,X: \ep S}} && {\bbrkt{E}} \\
  {\bbrkt{\Ga,X:S,Y:K}} && {\bbrkt{\Ga,X:S}} && {\bbrkt{\Ga}}
  \arrow["{\bbrkt{ \ep :: X ::Y}}"', from=1-1, to=2-1]
  \arrow["{\bbrkt{( \ep :: X )K}}", from=1-1, to=1-3]
  \arrow["{\bbrkt{K}}"', from=2-1, to=2-3]
  \arrow["{\bbrkt{ \ep S}}", from=1-3, to=1-5]
  \arrow["{\bbrkt{S}}"', from=2-3, to=2-5]
  \arrow["{\bbrkt{ \ep :: X }}"{description}, from=1-3, to=2-3]
  \arrow["{\bbrkt{\ep}}", from=1-5, to=2-5]
  \arrow["\lrcorner"{anchor=center, pos=0.125, rotate=0}, draw=none, from=1-3, to=2-5]
  \arrow["\lrcorner"{anchor=center, pos=0.125, rotate=0}, draw=none, from=1-1, to=2-3]
  \arrow["{\bbrkt{f X}}"', bend right, from=1-3, to=1-1]
  \arrow[dashed, from=1-3, to=2-1]
\end{cd}

The dashed arrow is then a morphism in the slice $\CC / \bbrkt{\Ga,X:S}$.
So we have (making the same argument the other way round and using $\be$ and $\eta$)
\[ \color{semantics}
  \begin{tikzcd}
  {\CC/\bbrkt{\CtxExt{\Ga}{X}{S}} \, (\bbrkt{ \ep :: X } , \bbrkt{K})} && {\CC/\bbrkt{\Ga}(\bbrkt{\ep},\bbrkt{{\PiCS{X}{S}{K}}})}
  \arrow["\textsc{pi-lam-obj}", shift left, from=1-1, to=1-3]
  \arrow["\textsc{pi-app-obj}", shift left, from=1-3, to=1-1]
\end{tikzcd} \]

This is closer to an adjunction: we can see that the left adjoint should be pullback along
$\bbrkt{S} : \bbrkt{\CtxExt{\Ga}{X}{S}} \to \bbrkt{\Ga}$.
\[ \bbrkt{ S }^* \bbrkt{\ep} = \bbrkt{ \ep :: X }\]
Now we make the right adjoint functorial as well.
We replace $\bbrkt{K}$ with a slice $\bbrkt{\de} : \bbrkt{\De} \to \bbrkt{\CtxExt{\Ga}{X}{S}}$.
% Could ask for this bijection to work for arbitrary pullback?
% The answer is \textit{not quite}.
% We can replace the context extension $\bbrkt{\CtxExt{\Ga}{X}{S} \to \bbrkt{\Ga}}$ with a context morphism
% $\bbrkt{\ObjJdg{\De}{\de}{\Ga}}$,
% but we will only be able to find a right adjoint for pullback along $\bbrkt{\de}$ when
% $\De$ \textit{only contains sorts}.
% This is because \textsc{pi-cls} restricts formation of function classes to classes dependent over sorts.

% \begin{mathpar}
%   \infer[\textsc{ctx-sort-emp}]{\CtxSortJdg{\bullet}}{}

%   \infer[\textsc{ctx-sort-ext}]{\CtxSortJdg{\CtxExt{\Ga}{X}{S}}}{\CtxSortJdg{\Ga} & \ObjJdg{\Ga}{S}{\SortCS}}
% \end{mathpar}

\[ \color{semantics}
\begin{tikzcd}
  {\CC/\bbrkt{\CtxExt{\Ga}{X}{S}} \, (\bbrkt{ S }^* \bbrkt{\ep} , \bbrkt{\de})} && {\CC/\bbrkt{\Ga}(\bbrkt{\ep},{\Pi_{\bbrkt{S}}\bbrkt{\de}})}
  \arrow["\textsc{pi-lam-obj}", shift left, from=1-1, to=1-3]
  \arrow["\textsc{pi-app-obj}", shift left, from=1-3, to=1-1]
\end{tikzcd}\]
Briefly and roughly: Write context $\De = V_{1} : \De_{1}, \cdots, V_{n} : \De_{n}$.
A slice morphism $O$ on the left consists of objects $\ObjJdg{\CtxExt{E}{X}{\ep S}}{O_{i}}{\Subst{O}{V}{\De_{i}}}$.
Then $\la$-abstraction gives objects $\ObjJdg{E}{\LamCS{X}{\ep S}{O_{i}}}{\PiCS{X}{\ep S}{\Subst{O}{V}{\De_{i}}}}$
on the right hand side.
So we should define ${\Pi_{\bbrkt{\de}}\bbrkt{\ze}}$ such that its $\CC$-object part is context
\[
  \bbrkt{V_{1} : \PiCS{X}{\ep S}{{\De_{0}}}, \cdots, V_{n} : {\PiCS{X}{\ep S}{{\De_{n}}}}}
\]
and its morphism part comes from each $Z_{i}$ being a class in context $\CtxExt{\Ga}{X}{S}$.
The proof that it is an adjunction is in the same vein as before.

\[
  \color{semantics}
  \bbrkt{S}^{*} \dashv \Pi_{\bbrkt{S}}\]

\subsection{Sorts (pre-equality)}
The category of contexts is \textit{not} a locally cartesian closed category.
In particular, we don't have pullbacks for arbitrary morphisms in the category.
However, (even for the fragment of the LF without the equality class)
we can find a subcategory that \textit{is} locally cartesian closed,
which essentially consists of context extentions, where the extensions only introduce sorts.
Furthermore, this subcategory $\SS_{\Pi}$ will be a \linkto{dfn_urm}{universe of representable maps}.

We know that for contexts extended with sorts $\bbrkt{S} : \bbrkt{\CtxExt{\Ga}{X}{S}} \to \bbrkt{\Ga}$
we have the adjunction $\bbrkt{S}^{*} \dashv \Pi_{\bbrkt{S}}$.
In particular $\bbrkt{\ElCS} : \bbrkt{\CtxExt{\ElCS : \SortCS}{X}{\ElCS}} \to \bbrkt{\ElCS : \SortCS}$
is such an extension, and there is a correspondence between pullbacks of $\bbrkt{\ElCS}$
and sort-judgments

\begin{multicols}{2}
  \[\color{syntax} \ObjJdg{\Ga}{S}{\SortCS}\]

  \[ \color{semantics}
    \begin{tikzcd}
      {\bbrkt{\Ga,S}} & {\bbrkt{\ElCS : \SortCS,X : \ElCS}} \\
      {\bbrkt{\Ga}} & {\bbrkt{\ElCS : \SortCS}}
      \arrow[from=1-1, to=1-2]
      \arrow["{\bbrkt{\ElCS}}", from=1-2, to=2-2]
      \arrow[from=2-1, to=2-2]
      \arrow["\bbrkt{S}"', from=1-1, to=2-1]
      \arrow["\lrcorner"{anchor=center, pos=0.125}, draw=none, from=1-1, to=2-2]
    \end{tikzcd}
    \]
\end{multicols}

Let us try to generate from $\bbrkt{\ElCS}$,
the smallest LCCC containing it (supposing it exists\footnote{
  This is reasonable given that we already have pullbacks and products
  for this class of maps.
}).
So we throw $\ElCS$ into $\SS_{\Pi}$,
throw in identities,
take the closure under composition,
and closure under pullback against $\CC$-morphisms.
After checking some examples,
we guess that $\SS_{\Pi}$ should be characterized as the set of
compositions of sort-context-exentions,
i.e. $\bbrkt{S_{1}} \circ \cdots \circ \bbrkt{S_{n}}$
(the identity on $\bbrkt{\Ga}$ when $n = 0$)
whenever
\[ \CtxJdg{\Ga} \quad \quad \ObjJdg{\Ga}{S_{1}}{\SortCS} \quad \quad \ObjJdg{\CtxExt{\Ga}{X_{1}}{S_{1}}}{S_{2}}{\SortCS} \quad \quad
   \cdots \quad \quad \ObjJdg{\CtxExt{\CtxExt{\Ga}{X_{1}}{S_{1}}, \cdots }{X_{n-1}}{S_{n-1}}}{S_{n}}{\SortCS}\]
To ensure these two characterizations of $\SS_{\Pi}$ are equivalent
it suffices to check that the second is stable under pullbacks, and closed under $\Pi$s.

To check stability under pullback along $\bbrkt{\si} : \bbrkt{\De} \to \bbrkt{\Ga}$,
first recall that for any sort $\bbrkt{S} : \bbrkt{\CtxExt{\Ga}{X}{S}} \to \bbrkt{\Ga}$,
the pullback is the substition on the sort $\bbrkt{\si^{*}S} : \bbrkt{\CtxExt{\De}{X}{\si^{*}S}} \to \bbrkt{\De}$
which is another sort.
This means that the pullback of any composition $\bbrkt{S_{1}} \circ \cdots \circ \bbrkt{S_{n}}$
is just the composition of the pullbacks (``pasting''),
and given that each piece of the pullback is a sort,
the composition is in $\SS_{\Pi}$.
For the case when $n = 0$ note that the pullback of the identity is the identity.

Closure under $\Pi$s: by stability under pullback along general context substitutions,
we know that for any morphism $g = \bbrkt{S_{1}} \circ \cdots \circ \bbrkt{S_{n}} \in \SS_{\Pi}$,
$g^{*}$ is a well-defined functor.
Furthermore, for each $\bbrkt{S_{i}}$ we have an adjunction $\bbrkt{S_{i}}^{*} \dashv \Pi_{\bbrkt{S_{i}}}$,
and the composition of adjunctions % ref?
yields an adjunction
\[\bbrkt{S_{n}}^{*} \circ \cdots \circ \bbrkt{S_{1}}^{*}=
  \bbrkt{g}^{*} \dashv \Pi_{\bbrkt{g}} = \Pi_{\bbrkt{S_{1}}} \circ \cdots \circ \Pi_{\bbrkt{S_{n}}}\]

Thus the two charactizations of $\SS_{\Pi}$ are equivalent and
$\SS_{\Pi}$ is a URM.

\subsection{Equality classes}
We showed that pullbacks of context morphisms along context extensions/classes
corresponds to substitution on classes.
Do we have pullback along any context morphism?
The answer is \textit{not in this type theory}: in order to construct the pullback,
we need an equality class, but our equality classes are restricted to $\SortCS$.
Let us temporarily throw in an extra rule for the sake of exposition,
and use it to construct a pullback.\footnote{
  This rule is included in Uemura's logical framework,
  and adding it corresponds to having representable map
  categories for semantics \cite{uemura2019general}.
}

\begin{multicols}{2}
    \[
  \color{syntax}{
        \infer[\textsc{eq-cls}]{\ClsJdg{\Ga}{\EqCS{K}{O_0}{O_1}}}
        {\ObjJdg{\Ga}{O_0}{K} & \ObjJdg{\Ga}{O_1}{K}}
    }
      \]
% https://q.uiver.app/?q=WzAsNCxbMCwwLCJcXERlIFxcdGltZXNfe1xcR2F9IEUiXSxbMiwwLCJFIl0sWzAsMiwiXFxEZSJdLFsyLDIsIlxcR2EiXSxbMCwxXSxbMiwzLCJcXGRlIiwyXSxbMSwzLCJlIl0sWzAsMl1d
{\color{semantics}\begin{cd}
  {\bbrkt{\De} \times_{\bbrkt{\Ga}} \bbrkt{E}} && \bbrkt{E} \\
  \\
  \bbrkt{\De} && \bbrkt{\Ga}
  \arrow["\pi_{E}", from=1-1, to=1-3]
  \arrow["\bbrkt{\de}"', from=3-1, to=3-3]
  \arrow["\bbrkt{\ep}", from=1-3, to=3-3]
  \arrow["\pi_{\De}"', from=1-1, to=3-1]
  \arrow["\lrcorner"{anchor=center, pos=0.125}, draw=none, from=1-1, to=3-3]
\end{cd}}
\end{multicols}

Again imagining $\De$ and $E$ as closed classes,
the pullback should be ``pairs of objects in $\De \times E$'' that
are equal upon their substitutions into the objects given by $\de$ and $\ep$.
In set-theoretic notation (since we are inspired by pullback in $\SET$)
\[ \set{ (X,Y) \in \De \times E \st \de(X) = \ep(Y) }\]
Translating this idea into a context means that we need
\begin{align*}
  {\bbrkt{\De} \times_{\bbrkt{\Ga}} \bbrkt{E}} & = \bbrkt{X : \De, Y : E, H : \EqCS{\Ga}{\de}{\ep}}\\
  \pi_{\De} & = \bbrkt{ X } \\
  \pi_{E} & = \bbrkt{ Y }
\end{align*}

In detail: if $\De = X_{1} : \De_{1}, \cdots , X_{n} : \De_{n}$
and $E = Y_{1} : E_{1}, \cdots, Y_{m} : E_{m}$
and $\Ga = Z_{1} : \Ga_{1}, \cdots , Z_{l} : \Ga_{l}$
then the context is
\[X_{1} : \De_{1}, \cdots , X_{n} : \De_{n}, Y_{1} : E_{1}, \cdots, Y_{m} : E_{m} ,
  H_{1} : \EqCS{\Ga_{1}}{\de_{1}}{\ep_{1}}, \cdots , H_{k} : \EqCS{\Ga_{l}}{\de_{l}}{\ep_{l}}\]
and the context morphisms are those that return the relevant variables from this context.

To show the universal property, given cone $\bbrkt{Z}$
we use that judgmental equalities
$\IdObjJdg{Z}{\ze \gg \de}{\ze \gg \ep}{\Ga}$
lift
to $\ObjJdg{Z}{\SelfCS}{\EqCS{\Ga}{\ze \gg \de}{\ze \gg \ep}}$ for existence
(this fact follows from judgmental equality being a congruence);
we use \textsc{unicity} for uniqueness.

Hence we conclude that adding the rule \textsc{eq-cls}
then would mean that the whole context category has pullbacks.

% \begin{multicols}{2}
%     \[
%   \color{syntax}{
%         \infer[\textsc{eq-cls}]{\ClsJdg{\Ga}{\EqCS{K}{O_0}{O_1}}}
%         {\ObjJdg{\Ga}{O_0}{K} & \ObjJdg{\Ga}{O_1}{K}}
%     }
%       \]


%   \color{semantics}{\[\bbrkt{\De} \times_{\bbrkt{\Ga}} \bbrkt{E}\]}
% \end{multicols}

\subsection{Sorts (post-equality)}
We want the corresponding semantics for the rule \textsc{eq-sort}.
This should give us some URM that is closed under equality-formation.
The situation is similar to \textsc{eq-cls},
except our assumption should be that we have
\[\begin{tikzcd}
  {\bbrkt{Y : E, X : \De, H : \EqCS{\Ga}{\de}{\ep}}} && \bbrkt{E} \\
  \\
  \bbrkt{\De} && \bbrkt{\Ga}
  \arrow["\pi_{E}", from=1-1, to=1-3]
  \arrow["\bbrkt{\de}"', from=3-1, to=3-3]
  \arrow["\bbrkt{\ep}", from=1-3, to=3-3]
  \arrow["\pi_{\De}"', from=1-1, to=3-1]
  \arrow["\lrcorner"{anchor=center, pos=0.125}, draw=none, from=1-1, to=3-3]
\end{tikzcd}\]
when $\Ga$ only contains sorts.
This guarantees that each equality class formed $\EqCS{\Ga_{i}}{\de_{i}}{\ep_{i}}$
is again a sort.

Now, ensuring $\Ga$ is sort-only,
we naively try to add $\bbrkt{\de} : \bbrkt{\De} \to \bbrkt{\Ga}$ into $\SS_{\Pi}$,
and try to make this a URM.
Firstly this means that $\pi_{E} = \bbrkt{\ep}^{*}\bbrkt{\de}$ should be in this set.
Although we can form the pullback along $\pi_{E}$ against any $\CC$-morphism
(it is a series of context extensions),
we do not get an adjunction $\pi_{E}^{*} \dashv \Pi_{\pi_{E}}$.
Having the adjunction requires an impredicative dependent function type,
i.e. one that has general classes as hypotheses.

However, if we require that \textit{both $\De$ and $\Ga$} are sort only,
then $\pi_{E}$ is a context extension of $E$ extended with only sorts,
so it is in $\SS_{\Pi}$ already.
So let us characterize our new candidate URM as
\[\SS_{\Pi,=} := \SS_{\Pi} \cup \set{ \bbrkt{\ObjJdg{\De}{\de}{\Ga}} \st \De,\Ga \text{ sort-only }}\]

$\SS_{\Pi,=}$ is stable under pullback, as noted above.
It is clearly closed under identity.
To show it is closed under composition,
we case on which part of the union the parts came from;
each case is easy.
To show it is closed under $\Pi$s, we just need to generalize the adjunction from before:

\begin{cd}
  {\CC/\bbrkt{\De} \, (\bbrkt{ \de }^* \bbrkt{\ep} , \bbrkt{\ze})} && {\CC/\bbrkt{\Ga}(\bbrkt{\ep},{\Pi_{\bbrkt{\de}}\bbrkt{\ze}})}
  \arrow["\textsc{pi-lam-obj}", shift left, from=1-1, to=1-3]
  \arrow["\textsc{pi-app-obj}", shift left, from=1-3, to=1-1]
\end{cd}

Briefly and roughly, if $\bbrkt{\ze} : \bbrkt{Z} \to \bbrkt{\De}$ for
$Z = V_{1} : Z_{1}, \cdots, V_{n} : Z_{n}$
then the $\CC$-object part of ${\Pi_{\bbrkt{\de}}\bbrkt{\ze}}$ is
\[
  \bbrkt{V_{1} : \PiCS{X}{\De}{\PiCS{Y}{\EqCS{\Ga}{\ep}{\de}}{Z_{1}},\cdots, V_{n} : \PiCS{X}{\De}{\PiCS{Y}{\EqCS{\Ga}{\ep}{\de}}{Z_{n}}}}}
\]
This can be formed since $\De$ is sort-only and $\Ga$ is sort only (hence the equality classes are sorts).
Furthermore, if $\bbrkt{\ze} \in \SS_{\Pi,=}$ then either it is a sort-context-extension of sort-only $\De$,
hence $Z$ is sort-only, or it $Z$ is sort-only by assumption.
Thus $\Pi_{\de}\bbrkt{\ze}$ would be sort-only by \textsc{pi-sort}.
So $\SS_{\Pi,=}$ is closed under $\Pi$s.
This concludes that $\SS_{\Pi,=}$ is a URM.

Note that $\SS_{\Pi}$ as previously described
is still a URM when we have equality as part of the syntax.
So now we have two URMs $\SS_{\Pi} \hookr \SS_{\Pi,=}$.

\subsection{Signatures}

Given a signature $\Si$, we can simply take the slice over
$\bbrkt{\Si}$ in the semantics to obtain semantics for the type theory generated by $\Si$.
Since a slice of a locally closed cartesian category is locally cartesian closed,
$\SS_{\Pi}/\bbrkt{\Si}$ and $\SS_{\Pi,=}/\bbrkt{\Si}$ are still LCCC.

\subsection{Classifying context extension}
\link{dfn_universe}

Let us briefly return to studying $\SS_{\Pi}$.
In the original definition of a universe in a topos \cite{streicher},
property $2$ includes all monos from $\CC$ (in particular identity morphisms)
into the universe.
Including all monos corresponds to including all proposition-context-extensions,
i.e. $\CtxExt{\Ga}{X}{P}$ where $P$ is a proposition.

The original property $5$ said that all morphisms of $\SS_{\Pi}$ are pullbacks of a single morphism
$\ElCS : E \to U$.
In our case this should be the map
$\bbrkt{S} : \bbrkt{\CtxExt{S : \SortCS}{X}{S}} \to \bbrkt{S : \SortCS}$.
This would imply in particular that composition and identity are also certain pullbacks of $\brkt{\ElCS}$.
In the case of composition, this amounts to requiring dependent pair sorts (aka $\Si$-types) in the type theory.
In the case of identity, this amounts to requiring a unit sort, i.e. a sort containing exactly one object.
